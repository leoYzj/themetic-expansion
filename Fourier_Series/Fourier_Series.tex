\documentclass{ctexart}

\usepackage{silence}
\usepackage[top=2cm, bottom=2cm, left=2.5cm, right=2.5cm]{geometry}
\usepackage{graphicx}
\usepackage{grffile}
\graphicspath{{D:/code/Math_behind_Signal_and_System/Figures}}
\usepackage{amsmath,amssymb,amsfonts}
\usepackage{multicol}
\usepackage{varwidth}
\usepackage{dsfont}
\usepackage{hyperref}
\usepackage{fix-cm}
\usepackage{bm}
\usepackage{xcolor}
\usepackage{array}
\usepackage{booktabs}
\usepackage{float}
\usepackage{pifont}
\usepackage{enumitem}
\usepackage{tikz}
\usepackage{hyperref}
\usetikzlibrary{shapes,arrows,positioning,calc}
\usepackage{silence}
\usetikzlibrary{calc}
\ActivateWarningFilters
\WarningFilter{latex}{Font shape}
\WarningFilter{latex}{Some font shapes}
\vfuzz=100pt  % 垂直方向容忍度
\hfuzz=100pt  % 水平方向容忍度  
\vbadness=10000
\hbadness=10000
\overfullrule=0pt  % 不标记过满行
\allowdisplaybreaks
\raggedbottom

\newcommand{\shah}{\operatorname{III}}
% 便捷命令:在文中书写原函数在上下限的取值,例如 \evalat{F(x)}{a}{b} 输出为 \left.F(x)\right|_{a}^{b}
\newcommand{\evalat}[3]{\left.#1\right|_{#2}^{#3}}
\newcommand{\lr}[2]{%
    \begin{center}
    \begin{minipage}[t]{0.45\textwidth}
        \centering
        \allowdisplaybreaks
        \textcolor{blue}{%
            \begin{varwidth}{\linewidth}
            $\begin{aligned}
            #1
            \end{aligned}$
            \end{varwidth}
        }
    \end{minipage}
    \hfill
    \begin{minipage}[t]{0.45\textwidth}
        \centering
        \allowdisplaybreaks
        \textcolor{red}{%
            \begin{varwidth}{\linewidth}
            $\begin{aligned}
            #2
            \end{aligned}$
            \end{varwidth}
        }
    \end{minipage}
\end{center}
}
\newlist{circlist}{enumerate}{1}
\setlist[circlist]{
    label=\protect\ding{\numexpr171+\arabic*\relax},
    left=0pt
}

\begin{document}

\section{勒贝格积分}\label{sec:Lebesgue}
狄利克雷函数
\[D(x)=
    \begin{cases}
        1 & x \in \mathbb{Q}    \\
        0 & x \in \mathbb{Q} ^C
    \end{cases}\]
(其中$\mathbb{Q}$ 表示有理数集,上标C在不引起歧义的前提下用来表示取补集)在
黎曼积分(也就是数学分析或高等数学课程,以及多数工科课程中用的积分)的意义下
是不可积的,因为它在所有点不连续,但是这样的函数在\textbf{勒贝格积分}的意义下可积,读
者可以这样理解:勒贝格积分考察函数在某个值处的“区间长度”(严格来讲应为测度)
,狄利克雷函数在一个有限区间上取值1的长度为0,因为有理数是可数集,取值0的长度
就是区间长度,所以其积分值为0。勒贝格积分是黎曼积分的推广,对于非反常积分,黎曼可积的函数一定
是勒贝格可积的,并且在同样的区间上积分值相等。对于勒贝格不可积的函数,一般不
在本课程的讨论范围。尽管我们很少接触只在勒贝格意义下才可积的函数,但后面我们将
逐步认识到勒贝格积分在傅里叶分析中的重要地位,读者应当对其有一个初步的认识。

\section{线性空间,正交基}\label{sec:Linear_Space}
首先介绍一些本章需可能用到的概念:度量公理、范数公理、内积公理、正交基、无穷维线
性空间和$L^p$空间,对其不感兴趣的读者,可以等需要时再阅读此小节的内
容,只要知道函数空间上的正交基是怎么回事即可。

在$\mathbb{C}$ -线性空间V中,如果定义了运算
$d(\cdot,\cdot):V\times V\rightarrow \mathbb{C}  $,满足\textbf{度量公理}:
\begin{enumerate}
    \item \textbf{非负性}:$d(x, y) \geq 0$
    \item \textbf{同一性}:$d(x, y) = 0$ 当且仅当 $x = y$
    \item \textbf{对称性}:$d(x, y) = d(y, x)$
    \item \textbf{三角不等式}:$d(x, z) \leq d(x, y) + d(y, z)$
\end{enumerate}
则称在V上定义了一种度量(事实上只要在拓扑空间中就能够定义度量)。

如果定义了运算
$\|\cdot \| :V\rightarrow \mathbb{C} $,满足\textbf{范数公理}
\begin{enumerate}
    \item \textbf{非负性}:$\|\mathbf{x}\| \geq 0$
    \item \textbf{同一性}:$\|\mathbf{x}\| = 0$ 当且仅当 $\mathbf{x} = \mathbf{0}$
    \item \textbf{齐次性}:$\|\alpha \mathbf{x}\| = |\alpha| \, \|\mathbf{x}\|$
    \item \textbf{三角不等式/次可加性}:$\|\mathbf{x} + \mathbf{y}\| \leq \|\mathbf{x}\| + \|\mathbf{y}\|$
\end{enumerate}
则称在V上定义了一种范数,称V为线性赋范空间,在线性赋范空间上可以定义极限:
\[\lim_{x \to x_0} f(x)=A:=\forall \epsilon>0\exists \delta>0(|x-x_0|<\delta\Rightarrow |f(x)-f(x_0)|<\epsilon) \]
这就是数学分析中的极限定义,只是将绝对值改成了范数,其余类似的极限定义不再赘述。
如果线性赋范空间中所有柯西列都有极限,那么这个空间称为完备的线性赋范空间,或称
为巴拿赫空间(有限维)、希尔伯特空间(无限维)。

\noindent 例1.1.连续函数空间$C[a,b]$是不完备的线性赋范空间:在1-范数
$\|f\|_1 =\int_{a}^{b}f(t)\,dt$下,其中的柯西列\begin{align*}
    f_n(t)=\begin{cases}
        0,&\text{if }-1\leq t <0\\
        nt,&\text{if }0\leq t <\frac{1}{n}\\
        1,&\text{if }\frac{1}{n}\leq t \leq 1
    \end{cases}
\end{align*}
的极限为单位阶跃函数$u(t)$(在0处取值为0),不属于$C[-1,1]$.

如果定义了运算
$\langle \cdot,\cdot\rangle:V\times V\rightarrow \mathbb{C} $,满足\textbf{内积公理}
\begin{enumerate}
    \item \textbf{正定性}:$\langle \mathbf{v}, \mathbf{v} \rangle \geq 0$ 且 $\langle \mathbf{v}, \mathbf{v} \rangle = 0$ 当且仅当 $\mathbf{v} = \mathbf{0}$
    \item \textbf{共轭对称性}:$\langle \mathbf{v}, \mathbf{w} \rangle = \overline{\langle \mathbf{w}, \mathbf{v} \rangle}$
    \item \textbf{第一变元的线性性}:
          \begin{itemize}
              \item \textbf{齐性}:$\langle \alpha \mathbf{v}, \mathbf{w} \rangle = \alpha \langle \mathbf{v}, \mathbf{w} \rangle$
              \item \textbf{可加性}:$\langle \mathbf{v} + \mathbf{w}, \mathbf{u} \rangle = \langle \mathbf{v}, \mathbf{u} \rangle + \langle \mathbf{w}, \mathbf{u} \rangle$
          \end{itemize}
\end{enumerate}
则称在V上定义了一种内积 (inner product),称V为内积空间。在内积空间上有著名的
柯西-施瓦兹不等式 (Cauchy-Shwartz inequality):
\[|\langle \mathbf{a,b}\rangle| \leq \| \mathbf{a}\| \| \mathbf{b}\| \]
它有一个经典的证明方法:不妨设$\mathbf{b}$不是零向量,任取$t\in \mathbb{R}$,有
\[0\leq \|\mathbf{a}+t\mathbf{b}\|^2=\|\mathbf{a}\|^2+2t\langle\mathbf{a,b}\rangle +t^2\|\mathbf{b}\|^2\]
令$t=-\frac{\langle\mathbf{a,b}\rangle}{\|\mathbf{b}\|^2}$,即得
\[0\leq \|\mathbf{a}\|^2-2\frac{\langle\mathbf{a,b}\rangle^2}{\|\mathbf{b}\|^2} +\frac{\langle\mathbf{a,b}\rangle^2}{\|\mathbf{b}\|^4}\|\mathbf{b}\|^2=\|\mathbf{a}\|^2-\frac{\langle\mathbf{a,b}\rangle^2}{\|\mathbf{b}\|^2}\]
这与要证明的不等式是等价的。有了柯西-施瓦兹不等式,三角不等式就是显然的了,这里
仅给出其表述,读者可以自行证明:
\[\forall \mathbf{a,b}\in V,\|\mathbf{a+b}\|\leq\|\mathbf{a}\|+\|\mathbf{b}\|\]

不难发现,只要取
$\|\mathbf{v}\|^2=\langle \mathbf{v}, \mathbf{v} \rangle$,
就由内积导出了一种范数,并且这种范数具有比一般的范数更强的性质;
只要取$d(\mathbf{x},\mathbf{y})=\|\mathbf{x-y}\|$,就由范数导出了一种度量,
并且这种度量具有比一般的度量更强的性质。

\textbf{正交基} (othorgnal bases)指的是内积空间V的一组基$\{\mathbf{v_1},\mathbf{v_2},\dots ,\mathbf{v_n}\}$
,满足$\langle \mathbf{v_i},\mathbf{v_j}\rangle =0,i \neq j$ ,对V中任一向量$\mathbf{w}$,
\[\mathbf{w}=\sum_{i = 1}^{n}  c_i \mathbf{v_i}\]
等式两边同时对$\mathbf{v_j}$做内积,得到
\begin{align}
    \langle \mathbf{w},\mathbf{v_j} \rangle=\langle \mathbf{v_j},\sum_{i = 1}^{n}  c_i \mathbf{v_i} \rangle =c_j\langle \mathbf{v_j},\mathbf{v_j} \rangle \\
    c_j=\frac{\langle \mathbf{w},\mathbf{v_j} \rangle}{\langle \mathbf{v_j},\mathbf{v_j} \rangle},
    \mathbf{w}=\sum_{j = 1}^{n}  \frac{\langle \mathbf{w},\mathbf{v_j} \rangle}{\langle \mathbf{v_j},\mathbf{v_j} \rangle} \mathbf{v_j}\label{eq:2.2}
\end{align}
如果$\{\mathbf{v_1},\mathbf{v_2},\dots ,\mathbf{v_n}\}$还满足
$\langle \mathbf{v_i},\mathbf{v_i}\rangle =1,i \in \{1,2,\dots ,n\}$,称这组基是
\textbf{标准正交基} (othornormal bases),此时空间中任意向量均有分解式
\begin{align}\label{eq:2.3}
    \mathbf{w}=\sum_{i=1}^{n}\langle \mathbf{w},\mathbf{v_i}\rangle \mathbf{v_i}
\end{align}
并且
\begin{align}
    |\mathbf{w}|^2 & =\langle \sum_{i=1}^{n}\langle \mathbf{w},\mathbf{v_i}\rangle \mathbf{v_i},\sum_{i=1}^{n}\langle \mathbf{w},\mathbf{v_i}\rangle \mathbf{v_i}\rangle                                           \\
                   & =\sum_{i=1}^{n}|\langle\mathbf{w,v_i}\rangle|^2\langle\mathbf{v_i,v_i}\rangle+\sum_{1\leq i<j\leq n}\langle \mathbf{w,v_i}\rangle\langle \mathbf{w,v_j}\rangle\langle \mathbf{v_i,v_j}\rangle \\
                   & =\sum_{i=1}^{n}|\langle\mathbf{w,v_i}\rangle|^2|\mathbf{v_i}|^2=\sum_{i=1}^{n}|\langle\mathbf{w,v_i}\rangle|^2\label{eq:2.6}
\end{align}
这正是高维情况下的勾股定理(毕达哥拉斯恒等式)。可见做正交基分解能够极大地简化
对线性赋范空间的研究。

对于\textbf{无限维线性空间}V,我们称向量列
$\lbrace\mathbf{v_i}\rbrace_{i=1}^{\infty}$是V的一组基,如果
\begin{itemize}
    \item \raggedright{} 线性无关性:任取基中的有限个向量,它们是线性无关的\\
    \item 有限生成性:任取向量$\mathbf{w} \in V$,存在有限个向量
          $V'=\lbrace\mathbf{v_1,v_2,\dots,v_r}\rbrace\subset V $,$\mathbf{w}$可以用$V'$线性表出
\end{itemize}
这里要求“有限”是为了避免敛散性的问题:例如收敛的级数构成线性空间,如果我们声
称取定了一组基(当然是无限的),并考察其中无限个基张成的空间,那么对于构成级数
的每一项,均需要考察其敛散性。然而,级数的敛散性自然可以对前有限项不做要求,它
们求和很可能不会收敛;从另一个角度来讲,一些更加抽象的线性空间中,也说不清楚基
的无限和是否收敛,甚至在没有范数的线性空间中无法定义收敛。

函数空间是一种典型的无限维线性空间(因为多项式空间已经是无限维的),根据学习线
性代数的经验,我们希望能找到一组单位正交基,使得函数在这组正交基下的分解能够体
现函数的某些性质并便于后续的运算,然而,
若只考虑有限和,这种想法所能研究的函数十分有限,例如我们马上就会见到的三角函数
系和指数函数系,它们作为无限阶可微函数,有限和也是无限阶可微的。所以,我们应考
虑将函数$f(t)$分解为一组相互正交的函数系$\{f_i(t)\}_{i=1}^{\infty}$组成的函
数项级数。

我们面临的另一个问题是如何在函数空间上定义内积,从而定义正交性。一种比较自然的
想法是利用(勒贝格)积分,积分区间有限,对于周期函数,自然地取为一个周期。换言
之,我们考虑在空间
\[L^2([0,T]):=\{f:[0,T]\rightarrow \mathbb{C} \mid \int_{T}|f(t)|^2\,dt<\infty\}\]
上定义内积(为了区别于分布的符号,这里内积用圆括号表示,$^*$表示取共轭):
\[(f,g):=\int_{T}f(t)g^*(t)\,dt \]
我们对这个定义做一些说明,但不给出证明,因为证明需要首先建立勒贝格积分的体系,
读者可借助黎曼积分直观地理解它们:\\
1.$L^p([0,T])(0<p\leq \infty)$空间表示在区间[0,T]上p次勒贝格可积的函数组成的函数空间,即
\[L^p([0,T]):=\{f:[0,T]\rightarrow \mathbb{C} \mid \int_{T}f^p(t)\,dt<\infty\}\]
$L^p([0,T])$具有性质:
\begin{itemize}
    \item \raggedright{} $L^p([0,T])$是线性空间\\
    \item 当$1\leq p \leq \infty$时,$L^p([0,T])$是线性赋范空间,
          $\| f \|_{p} := \bigl( \int_{T} |f(t)|^p \, dt \bigr)^{1/p}$,
          称之为$L^p$范数,次可加性由闵可夫斯基不等式保证
\end{itemize}
2.要求$f(t)$平方可积是为了保证$(f,f)=\int_{T}|f(t)|^2\,dt<\infty$,$f(t)$平
方可积能够推出$f(t)$是绝对可积的,从而是可积的(有限区间I上有$L^p(I)\supset  L^q(I),p<q$,无限区间上它们互不包含)\\
3.尽管对函数空间做了一些限制,我们研究的范围依旧是足够大的,闭区间上的平方可积
是一个比较弱的条件\\
4.柯西-施瓦兹不等式和三角不等式(它是闵可夫斯基不等式的特例)自然成立,它们证
明的过程不涉及空间的维数是否有限。

有了内积就可以定义范数,从而可以给出$L^2([0,T])$空间上的函数项级数的(依范数)
收敛的定义:如果
\[\lim_{n \to \infty} \| f(t)-\sum_{i = 1}^{n}  a_i f_i(t)\|=0\]
就认为级数$\sum_{i = 1}^{n}  a_i f_i(t)$是$f(t)$在这个正交函数系下的分解,
此时记\[f\sim\sum_{n=1}^{\infty}a_n f_n\]
它并不意味着等式右侧的函数项级数在某一点收敛于f.在$L^2([0,T])$空间中,我们不
区分仅在零测集(“区间长度”的总和总能取到任意小正数,例如至多可数集)上不相等的
函数,换言之,$L^2([0,T])$空间不是常规意义下的函数的集合,而是\textbf{几乎处处}
(almost every,a.e.)相等的函数构成的等价类,这里的等号表示的是两侧的函数同属一
个等价类,至于逐点收敛、一致收敛性,需要另作讨论。

可以想象,依范数收敛要求极限内的函数相当接近于0,但如果在一个点处产生了误差,不论
误差多大,都不会影响积分的值。事实上,只要存在误差的点构成零测集,就不会影响积分的值,这时我们称
$\sum_{i = 1}^{n}  a_i f_i(t)$几乎处处收敛
于$f(t)$,只是这样弱的要求有时会导致积分在黎曼积分的意义下不存在,但勒贝格积分
可以处理这种情况,读者可以参考\ref{sec:Lebesgue}连续信号与离散信号中对勒贝格积分
的讨论。

如果不存在非零的函数$g(t)\notin\{f_i(t)\}_{i=1}^{\infty}$使得$g(t)$与
$\{f_i(t)\}_{i=1}^{\infty}$中的所有函数正交,我们称$\{f_i(t)\}_{i=1}^{\infty}$
为\textbf{完备正交函数系},这意味着$L^2([0,T])$空间中的任一函数$f(t)$均可分
解为这个函数系的函数项级数$\sum_{i = 1}^{\infty}  a_i f_i(t)$,由公式 (\ref{eq:2.2}),
\[a_i=\frac{(f,f_i)}{(f_i,f_i)}=\frac{\int_{T}f(t)f_i^*(t)\,dt}{\int_{T}|f_i(t)|^2\,dt}\]
细心的读者可能已经发现,这里得到的公式用到了有限维线性空间中的结论,但要推广到
无限维线性空间并不是显然的。我们将在下一节给出帕塞瓦尔定理之后一并讨论这个问题。

典型的标准完备正交函数集有贝塞尔 (Bessel)函数、勒让德 (Legendre)多项式、小
波 (wavelet)变换基函数等,下面仅讨论三角函数系和指数函数系。

\section{傅里叶级数}\label{sec:Fourier_Series}
首先回顾数学分析中几个计算傅里叶级数的公式。考虑将周期为T的函数f展开为
\begin{align*}
    f(t) & =\frac{a_0}{2}+\sum_{k = 1}^{\infty} a_k \cos(k\omega t)+b_k\sin(k\omega t) \\
         & =\frac{c_0}{2}+\sum_{k = 1}^{\infty} c_k\cos(k\omega t+\varphi _k)
\end{align*}
(其中$\omega =\frac{2\pi }{T}$为\textbf{基波角频率},$k\omega (k>1,k\in \mathbb{Z} )$
为k次\textbf{谐波角频率})则
\[a_k=\frac{2}{T}\int_T f(t)\cos(k\omega t)\,dt\]
\[b_k=\frac{2}{T}\int_T f(t)\sin(k\omega t)\,dt\]
\[c_k=\sqrt{a_k^2+b_k^2}\]

下面用完备标准正交函数系的观点来得到以上公式。在学习数学分析时,我们已经看到三
角函数系$1,\sin(\omega t),\cos(\omega t),\sin(2\omega t),\cos(2\omega t),\dots(\omega =\frac{2\pi}{T})$
是正交的(读者可以自行验证),但不是单位正交的,因为
\[(\sin(k\omega t),\sin(k\omega t))=\int_{T}\sin^2(k\omega t)\,dt=\int_{T}\frac{1-\cos(2k\omega t)}{2}=\frac{T}{2}\]
\[(\cos(k\omega t),\cos(k\omega t))=\int_{T}\cos^2(k\omega t)\,dt=\int_{T}\frac{1+\cos(2k\omega t)}{2}=\frac{T}{2}\]
可以将它们单位化,也可以直接采用公式 (\ref{eq:2.2}),
\[a_k=\frac{(f(t),\cos(k\omega t))}{(\cos(k\omega t),\cos(k\omega t))}=\frac{\int_{T}f(t)\cos(k\omega t)^*(t)\,dt}{\int_{T}|\cos(k\omega t)|^2\,dt}
    =\frac{2}{T}\int_{T}f(t)\cos(k\omega t)(t)\,dt\]
\[b_k=\frac{(f(t),\sin(k\omega t))}{(\sin(k\omega t),\sin(k\omega t))}=\frac{\int_{T}f(t)\sin(k\omega t)^*(t)\,dt}{\int_{T}|\sin(k\omega t)|^2\,dt}
    =\frac{2}{T}\int_{T}f(t)\sin(k\omega t)(t)\,dt\]

当$f(t)$为偶函数,或者由$f(t)$做偶延拓时,展开式为
\[f(t)=\frac{a_0}{2}+\sum_{k = 1}^{\infty} a_k\cos(k\omega t)\]
其中
\[a_k=\frac{4}{T}\int_{0}^{\frac{T}{2}} f(t)\cos(k\omega t)\,dt\]
\[b_k=0\]
当$f(t)$为奇函数,或者由$f(t)$做奇延拓时,展开式为
\[a_k=0\]
\[b_k=\frac{4}{T}\int_{0}^{\frac{T}{2}} f(t)\sin(k\omega t)\,dt\]
容易利用对称性得到这些公式。如果将傅里叶级数展开
式$f(t) =\frac{a_0}{2}+\sum_{k = 1}^{\infty} a_k \cos(k\omega t)+b_k\sin(k\omega t)$
写为
\begin{align*}
    f(t) = & \frac{a_0}{2}+\sum_{k = 1}^{\infty} a_k \cos(k\omega t) \\
           & +\sum_{k = 1}^{\infty} b_k \sin(k\omega t)
\end{align*}
则前半部分为偶函数,称之为$f(t)$的\textbf{偶分量}$f_e(t)$;后半部分为奇函数,称之为
$f(t)$的\textbf{奇分量}$f_o(t)$。高中数学中我们知道,函数的偶分量和奇分量都是唯一的,
并且\begin{align*}
    f_e(t)=\frac{f(t)+f(-t)}{2} \\
    f_o(t)=\frac{f(t)-f(-t)}{2}
\end{align*}

除了奇偶性,还可以从奇次谐波、偶次谐波的角度来理解函数。函数$f(t)$称为
\textbf{奇谐函数},如果后半个周期的函数是前半个周期的负镜像,即\begin{equation}
    f\left(t+\frac{T}{2}\right)=-f(t)
\end{equation}
这时函数的傅里叶级数展开只有奇次谐波分量:
\begin{align*}
    a_k =&\frac{2}{T}\int_{T}f(t)\cos(k\omega t)\,dt\\
    &=\frac{2}{T}\int_{-T/2}^{0}f(t)\cos(k\omega t)\,dt+\frac{2}{T}\int_{0}^{T/2}f(t)\cos(k\omega t)\,dt\\
    &=\frac{2}{T}\int_{0}^{T/2}\left(f(t)\cos(k\omega t)+f(t-T/2)\cos(k\omega(t-T/2))\right)\,dt\\
    &=(1-\cos(k\pi))\frac{2}{T}\int_{0}^{T/2}f(t)\cos(k\omega t)\,dt\\
    &=\begin{cases}
        0,&\text{if }k\text{为偶数}\\
        \frac{4}{T}\int_{0}^{T/2}f(t)\cos(k\omega t)\,dt,&\text{if }k\text{为奇数}
    \end{cases}
\end{align*}
同理有\begin{align*}
    b_k=\begin{cases}
        0,&\text{if }k\text{为偶数}\\
        \frac{4}{T}\int_{0}^{T/2}f(t)\sin(k\omega t)\,dt,&\text{if }k\text{为奇数}
    \end{cases}
\end{align*}
有了奇谐函数自然也能够定义偶谐函数:$f\left(t+\frac{T}{2}\right)=f(t)$,容
易验证它只有偶次谐波分量,但从定义式可以看出这只是周期减半的函数。

由欧拉公式$e^{ik\omega t}=\cos(k\omega t)+i\sin(k\omega t)$,得到
\[\cos(k\omega t)=\frac{e^{ik\omega t}+e^{-ik\omega t}}{2},\sin(k\omega t)=\frac{e^{ik\omega t}-e^{-ik\omega t}}{2i}\]
(特别地,$c_0=c_0^*\Rightarrow c_0\in\mathbb{R} $)
故函数$f(t)$也可在指数函数系下展开:
\[f(t)=\sum_{k = 0}^{\infty}  c_k e^{ik\omega t} ,c_k=\frac{a_k-ib_k}{2},c_{-k}=\frac{a_k+ib_k}{2}=c_k^*,k\in \mathbb{N}\]
$\{e^{ik\omega t}\}_{k=0}^{\infty}$是完备正交函数系,
\begin{align*}
    (e^{ik_1\omega t},e^{ik_2\omega t}) & =\int_{T}e^{ik_1\omega t}(e^{ik_2\omega t})^*\,dt                 \\
                                        & =\int_{T}e^{i(k_1-k_2)\omega t}                                   \\
                                        & =\frac{2}{i\omega (k_1-k_2)}\evalat{e^{i(k_1-k_2)\omega t}}{0}{T} \\
                                        & =0(k_1,k_2\in \mathbb{Z},k_1\neq k_2)                             \\
    (e^{ik\omega t},e^{ik\omega t})     & =\int_{T}e^{ik\omega t}(e^{ik\omega t})^*\,dt                     \\
                                        & =\int_{T}\,dt=T(k\in \mathbb{Z} )
\end{align*}
和三角函数系的情况一样,我们得到
\begin{align*}
    c_k & =\frac{(f(t),e^{ik\omega t})}{(e^{ik\omega t},e^{ik\omega t})} \\
        & =\frac{1}{T}\int_{T}f(t)(e^{ik\omega t})^*\,dt                 \\
        & =\frac{1}{T}\int_{T}f(t)e^{-ik\omega t}\,dt
\end{align*}
有时也将$c_k$记作$\hat{f}(k\omega)$或$\hat{F}(k)$,表示f在频域中的点$k\omega$处的值。一般而
言,我们只将最小正周期称为一个函数的周期,但周期为T的函数可以有多个频率
$k\omega(k \in \mathbb{Z})$,绘制频谱时,由于难以画出复数,常用\textbf{幅度谱}
$|\hat{f}(k\omega)|-\omega$和\textbf{相位谱}$\phi_k-\omega$来表征函数,其
中$\phi_k=\arg\hat{f}(k\omega)$。对于实信号,
\[c_k=(c_{-k})^*,|\hat{f}(k\omega)|=|\hat{f}(-k\omega)|,\phi_{-k}=-\phi_k\]
即幅度谱为偶函数,相位谱为奇函数,所以实信号的频谱中有一半是冗余的,按照展开式
\[\frac{c_0}{2}+\sum_{k = 1}^{\infty} c_k\cos(k\omega t+\varphi _k)\]绘制
的频谱$c_k-\omega$(注意不是指数函数形式的傅里叶系数)和$\phi_k-\omega$称为\textbf{单边频谱}
,而完整的频谱称为\textbf{双边频谱},从
\[\cos(k\omega t+\varphi _k)=\frac{e^{i(k\omega t+\varphi_k)}+e^{-i(k\omega t+\varphi_k)}}{2}\]
可知单边频谱相比双边频谱,在给定正频率处的幅值加倍,相位不变。这里的频率实际上是角频率$\omega$,用频率
f画频谱只涉及图像的横向伸缩,此处不再赘述。

需要指出的是,本小节中研究的函数均在$L^2([0,T])$空间中,但这并不能保证傅里叶
级数存在且收敛,保证这一点需要\textbf{狄利克雷条件}:
\begin{itemize}[nosep, left=0pt]
    \item $\int_{T}|f(t)|\,dt<\infty$
    \item 在一个周期内f连续或有有限个第一类间断点,即\textbf{分段连续} (piecewise continuous)
    \item 在一个周期内,f的极值点个数有限
\end{itemize}
满足此条件时,f的傅里叶级数展开在在任意点
收敛到其左右极限的平均值,这个结果称为\textbf{狄利克雷定理}。前两个条件是容易
理解的,对于最后一个条件,它实际上相当于要求f是有界变差函数 (Bounded Variatioin Function),
感兴趣的读者可以在实变函数的教材中了解这种函数。在\ref{sec:Asymptotic_Behaviour}中
我们将讨论另外的更易理解的条件。

下面考虑函数空间中的“勾股定理”。由公式 (\ref{eq:2.6}),
\[|f|^2=\sum_{k=1}^{\infty}c_k^2|e^{ik\omega t}|^2=T\sum_{k=1}^{\infty}c_k^2\]
即\[P=\frac{1}{T}\int_{T}|f(t)|^2\,dt=\sum_{k=1}^{\infty}c_k^2\]
这个公式称为\textbf{帕塞瓦尔定理} (Parseval's Thoerem)或\textbf{瑞利恒等式} (Rayleigh's Identity),
P为平均功率。

至此,我们得到了傅里叶系数的公式和帕塞瓦尔定理,但其实证明用到的结论是基于有限
维线性空间的,现在就来填补这个逻辑漏洞,对此不感兴趣的读者可以忽略这部分内容。
以下设$\{\phi_n\}_{n=1}^{\infty}$是$L^2(a,b)$的标准正交基,$f\in L^2(a,b)$
(注意这里已经不局限于讨论傅里叶级数,并且与前文未标准化的正交基略有形式上的差别)。

\textbf{引理2.1}:贝塞尔不等式 (Bessel's Inequality)\begin{align*}
    \sum_{n=1}^{\infty}|(f,\phi_n)|^2\leq\|f\|^2
\end{align*}
\textbf{Proof:}
\begin{flalign*}
     & \text{由勾股定理,}\|\sum_{n=1}^{N}(f,\phi_n)\phi_n\|^2= \sum_{n=1}^{N}(f,(f,\phi_n)\phi_n)=\sum_{n=1}^{N}\overline{(f,\phi_n)}(f,\phi_n)=\sum_{n=1}^{N}|(f,\phi_n)|^2         \\
     & \text{因此,对任意正整数N,}0\leq                            \|f-\sum_{n=1}^{N}(f,\phi_n)\phi_n\|                                                                                  \\
     & \hspace{4cm}=                                                  \|f\|^2-2Re(f,\sum_{n=1}^{N}(f,\phi_n)\phi_n)+\|\sum_{n=1}^{N}(f,\phi_n)\phi_n\|^2                        \\
     & \hspace{4cm}=                                                    \|f\|^2-2\sum_{n=1}^{N}|(f,\phi_n)|^2+\sum_{n=1}^{N}|(f,\phi_n)|^2=\|f\|^2-\sum_{n=1}^{N}|(f,\phi_n)|^2
\end{flalign*}
令$N\to\infty$即证。\\
从第二行到第三行用到了恒等式$\|\mathbf{a+b}\|^2=\|\mathbf{a}\|^2+2Re\langle\mathbf{a,b}\rangle+\|\mathbf{b}\|^2$,
Re表示取实部,这个结论十分简单,留予读者自证。在最终的结论帕塞瓦尔定理中这个不
等号将变成等号,但它是不可或缺的,并且我们还将在\ref{sec:Asymptotic_Behaviour}中见到它。

\textbf{引理2.2}:级数$\sum_{n=1}^{N}(f,\phi_n)\phi_n$依范数收敛,并且$\|\sum_{n=1}^{\infty}(f,\phi_n)\phi_n\|\leq\|f\|$\\
\textbf{Proof:}
\begin{flalign*}
     & \text{由贝塞尔不等式,}\sum_{n=1}^{\infty}|(f,\phi_n)|^2                    \leq\|f\|^2<\infty,n\to\infty\text{时}|(f,\phi_n)|\to 0                         \\
     & \text{任取}m_1,m_2\in\mathbb{N},m_1<m_2,\text{由勾股定理,}                 \|\sum_{n=m_1}^{m_2}(f,\phi_n)\phi_n\|^2=\sum_{n=m_1}^{m_2}|(f,\phi_n)|^2\to 0 \\
     & \text{因此}\sum_{n=1}^{\infty}(f,\phi_n)\phi_n\text{构成柯西列.}                                                                                          \\
     & \text{令}m_1=1,m_2\to\infty,\|\sum_{n=1}^{\infty}(f,\phi_n)\phi_n\| =\sum_{n=1}^{\infty}|(f,\phi_n)\phi_n|^2\leq\|f\|
\end{flalign*}
柯西列能够推出收敛是因为$L^2(a,b)$是无限维的完备度量空间,即\textbf{希尔伯特空间} (Hilbert space)
,见\ref{sec:Linear_Space}。构建这个引理是为了使用希尔伯特空间中内积的连续性,其表述
见下一个命题。

\textbf{命题2.3}:希尔伯特空间H中的内积具有连续性,即如果级数$\sum_{n=1}^{\infty}\phi_n$
的部分和$S_N$依范数收敛到S,则任给$y\in H$,总有
\[\lim_{N\to\infty}\langle S_n,y\rangle=\langle S,y\rangle\]
\textbf{Proof:}\begin{align*}
     & \langle S,y\rangle-\lim_{N\to\infty}\langle S_n,y\rangle=\lim_{N\to\infty}\langle S-S_n,y\rangle                                     \\
     & \lim_{N\to\infty}\|S-S_N\|=0\Rightarrow \lim_{N\to\infty}|\langle S-S_n,y\rangle|\leq\lim_{N\to\infty}\|S-S_N\|\|y\|=0               \\
     & \hspace{3cm}\Rightarrow \lim_{N\to\infty}\langle S-S_n,y\rangle=0\Rightarrow\lim_{N\to\infty}\langle S_n,y\rangle=\langle S,y\rangle
\end{align*}

\textbf{定理2.4}:以下三个命题是等价的:(对于符号$\sim$,参考\ref{sec:Linear_Space})
\begin{enumerate}
    \item $\forall n,(f,\phi_n)=0\Rightarrow f\sim 0$,即$\{\phi_n\}_{n=1}^{\infty}$是完备的标准正交基
    \item $\forall f\in L^2(a,b)$,有$f\sim\sum_{n=1}^{\infty}(f,\phi_n)\phi_n$
    \item $\forall f\in L^2(a,b)$,有\textbf{帕塞瓦尔恒等式}:
          \[\|f\|^2=\sum_{n=1}^{\infty}|(f,\phi_n)|^2\]
\end{enumerate}
\textbf{Proof:}
我们将证明$1\Rightarrow 2\Rightarrow 3\Rightarrow 1$.\\
$1\Rightarrow 2$:\begin{align*}
     & \text{令}g\sim f-\sum_{n=1}^{\infty}(f,\phi_n)\phi_n.                                                              \\
     & \forall m\in\mathbb{N},(g,\phi_m)=(f,\phi_m)-\sum_{n=1}^{\infty}(f,\phi_n)(\phi_n,\phi_m)=(f,\phi_m)-(f,\phi_m)=0 \\
\end{align*}
根据1知g=0,即2.这里内积与求和的换序是由命题2.3保证的。\\
$2\Rightarrow 3$:由勾股定理,
\[\|f\|^2=\lim_{N\to\infty}\|\sum_{n=1}^{N}(f,\phi_n)\phi_n\|^2=\lim_{N\to\infty}\sum_{n=1}^{N}|(f,\phi_n)|^2=\sum_{n=1}^{\infty}|(f,\phi_n)|^2\]
$3\Rightarrow 1$:$(f,\phi_n)=0\Rightarrow\|f\|=0\Rightarrow f\sim 0$.
\begin{figure}[H]
    \centering
    \includegraphics[width=0.6\textwidth]{Figure_2}\label{fig:2.1}
    \caption{周期矩形脉冲信号及其频谱}
\end{figure}
\noindent 例2.1.\textbf{周期矩形脉冲信号}的傅里叶级数展开和频谱图\\
脉冲宽度为$\tau$,脉冲幅度为E,周期为$T (\tau<T)$的周期矩形脉冲信号,基波角频率
$\omega=\frac{2\pi}{T}$,傅里叶级数展开为
\[f(t)=\sum_{k = 1}^{\infty}  \frac{E\tau}{T}Sa(\frac{k\omega \tau}{2})e^{ik\omega t}\]

如图\ref{fig:2.1},可以看到,这个频谱与取样函数$Sa(\omega)$非常相似(为了体现这一点,绘制频谱时将
基波角频率大幅减小,并不是第一张图直接做傅里叶级数展开的结果),原因将在傅里叶变换的章节
中给出。

\textbf{带宽} (bandwidth)指最高频率与最低频率之差,表征信号频率的集中程度。对
于实信号,有时仅考虑正频率,带宽减半。周期矩形脉冲信号的频谱是无限的,但能量基
本集中在最靠近y轴的两个零点之间,此时可以将带宽定义为\textbf{第一过零点带宽}
$B=\frac{2\pi}{\tau}$(仅考虑正频率)。

\section{*傅里叶级数的渐进特性,吉布斯现象}\label{sec:Asymptotic_Behaviour}

在用计算机模拟函数的傅里叶级数展开时,只能取有限项,自然要问计算到多少项时误差
足够小,为此,我们不加证明地给出以下定理:
\begin{align*}
     & \text{设}f\in C^p(\mathbb{R} )(p\geq1)\text{是周期函数,则部分和}
    S_N^f(t)=\sum_{-N}^{N}c_k e^{ik\omega t}
    \text{在}\mathbb{R} \text{上逐点收敛、}                                   \\
     & \text{内闭一致收敛,且} \max|f(t)-S_N^f(t)|<\frac{1}{N^{p-\frac{1}{2}}}
\end{align*}
其中$C^p(\mathbb{R} )$表示p次连续可导的函数集。

当$f(t)$不连续时,傅里叶级数的会在间断点处产生\textbf{吉布斯现象} (Gibbs' Phenomenon)
:部分和$S_N^f(t)$在间断点处总会\textbf{过冲}(在间断点两侧出现超过原函数的峰值)
,过冲幅度约为跳变幅度的9\%,并且$S_N^f(t)$会在间断点附近高频振荡,例如对于跳变
幅度为2、周期为$2\pi$的周期矩形脉冲信号
\[R(x) =
    \begin{cases}
        1  & \text{if } 0<x<\pi  \\
        -1 & \text{if } -\pi<x<0
    \end{cases}\]
其傅里叶级数的跳变值为1,
$\varlimsup_{N \to \infty}S_N^R(t)=1.089490 \dots$。
这是因为光滑的基函数很难逼近这种剧烈的局部变化,不得不用高频分量来补偿,高频分
量带来了剧烈震动。$\varlimsup_{N \to \infty}S_N^R(t)>1$并不意味着狄
利克雷定理失效,因为定理给出的是逐点收敛而非一致收敛,
\[\varlimsup_{N \to \infty}S_N^R(t)=\lim_{N \to \infty}\max_{t\in \mathbb{R} }S_N^R(t)\neq \max_{t\in\mathbb{R}}\lim_{N\to\infty}S_N^R(t)\]
\begin{figure}
    \centering
    \includegraphics[width=0.5\textwidth]{gibbs}
    \caption{吉布斯现象示意图}
\end{figure}

为了直观地理解它,我们来看一个经典的例子:
\begin{align*}
    f_n(x)=
    \begin{cases}
        nx   & \text{if } 0<x\leq 1/n        \\
        2-nx & \text{if } 1/n<x<2/n \\
        0    & \text{otherwise}
    \end{cases}
\end{align*}
随n增大,$f(x)$逐点趋于0,因为对每一点$2/n$总能取到更小的值;但$f(x)$
的最大值永远是1。

研究傅里叶级数的渐进特性时,一个非常好用的工具是\textbf{狄利克雷核} (Dirichlet kernel):
\[D_N(t)=\sum_{k=-N}^{N}e^{ik\omega t}=1+\sum_{k=1}^{N}(e^{ik\omega t}+e^{-ik\omega t})=1+2\sum_{k=1}^{N}\cos(k\omega t)\]
它是依赖于所研究函数的周期T的,但简便起见,在符号$D_N(t)$中不体现这一点。我们可
以用等比数列求和或积化和差裂项的方法化简$D_N(t)$:
\begin{align*}
    D_N(t) & =\sum_{k=-N}^{N}e^{ik\omega t}=e^{-iN\omega t}\frac{1-e^{i(2N+1)\omega t}}{1-e^{i\omega t}}                  \\
           & =\frac{e^{i(N+1)\omega t}-e^{-iN\omega t}}{e^{i\omega t}-1}                                   \\
    D_N(t) & =1+\sum_{k=1}^{N}(e^{ik\omega t}+e^{-ik\omega t})=1+2\sum_{k=1}^{N}\cos(k\omega t)           \\
           & =1+\frac{2}{\sin(\frac{\omega t}{2})}\sum_{k=1}^{N}\cos(k\omega t)\sin(\frac{\omega t}{2})                   \\
           & =1+\frac{1}{\sin(\frac{\omega t}{2})}\sum_{k=1}^{N}(\sin(k+\frac{1}{2})\omega t-\sin(k-\frac{1}{2})\omega t) \\
           & =1+\frac{\sin(N+\frac{1}{2})\omega t-\sin(\frac{\omega t}{2})}{\sin(\frac{\omega t}{2})}                     \\
           & =\frac{\sin(N+\frac{1}{2})\omega t}{\sin(\frac{\omega t}{2})}
\end{align*}
这两种结果是相符的,读者可自行验证,并且可以从后一结果想象出狄利克雷核的函数图
像,它被$\pm 1/\sin(\frac{\omega t}{2})$包络并高速振荡。将在“取样与插值”的章
节中介绍,$D_N(t)$是$\shah$函数的部分和,在$nT(n\in\mathbb{Z})$处,随$N\to\infty$,
D也趋于无穷,并在其他位置趋于0。这是又一个最大值不趋于0,但逐点趋于0的例子。函数图像如下。
\begin{figure}[H]
    \centering
    \includegraphics[width=0.7\textwidth]{Figure_3}
\end{figure}

引入狄利克雷核后,就可以用以下恒等式研究傅里叶级数的部分和:
\begin{align*}
    S_N^f(t) & =\sum_{-N}^{N}c_k e^{ik\omega t}                                                                                \\
             & =\sum_{-N}^{N}\left(\frac{1}{T}\int_{T}f(\tau)e^{-k\omega \tau}\,d\tau\right) e^{ik\omega t}                    \\
             & =\frac{1}{T}\int_{T}f(\tau)\sum_{k=-N}^{N}e^{ik\omega (t-\tau)}\,d\tau                                          \\
             & =\frac{1}{T}\int_{T}f(\tau)D_N(t-\tau)\,d\tau                                                                   \\
             & =\frac{1}{T}\int_{T}f(t-\tau)D_N(\tau)\,d\tau                                                & (\tau\to t-\tau) \\
             & =\frac{1}{T}\int_{T}f(t+\tau)D_N(\tau)\,d\tau                                                & (\tau\to t+\tau)
\end{align*}

在讨论傅里叶级数的收敛性前,先给出两个引理。第一个引理表明狄利克雷核在半周期上
积分值为$\frac{T}{2}$,在证明傅里叶级数的逐点收敛性时将用到它。
\begin{align}
    \int_{-\frac{T}{2}}^{0}D_N(t)\,dt=\int_{0}^{\frac{T}{2}}D_N(t)\,dt=\frac{T}{2}\label{eq:2.14}
\end{align}
\textbf{Proof:}
\begin{align*}
    D_N(t)&=1+2\sum_{k=1}^{N}\cos(k\omega t)\\
    \int_{0}^{\frac{T}{2}}D_N(t)\,dt & =\int_{0}^{\frac{T}{2}}\left(1+\sum_{k=1}^{N}\cos(k\omega t)\right)\,dt                              \\
                                     & =\frac{T}{2}+\sum_{k=1}^{N}\left.\frac{\sin(k\omega t)}{k\omega}\right|_{0}^{\frac{T}{2}} \\
                                     & =\frac{T}{2}+\frac{1}{\omega}\sum_{k=1}^{N}\frac{\sin(k\pi)}{k}=\frac{T}{2}               \\
    \intertext{$D_N(t)$是偶函数,得证。}
\end{align*}

第二个引理是\textbf{贝塞尔不等式} (Bessel's Inequality):设$ f\in L^2([0,T]),c_n=\frac{1}{T}\int_{T}f(t)e^{-ik\omega t}\,dt$,则
\begin{align}
    \sum_{-\infty}^{\infty}|c_n|^2\leq \frac{1}{T}\int_{T}|f(t)|^2\,dt\label{eq:2.15}
\end{align}
它给出了傅里叶系数平方和的上界的估计。收敛级数的通项必收敛,所以由此可以看出$c_n\to 0,n\to\infty$.
\begin{align*}
    \intertext{\textbf{Proof:}}
    |f(t)-\sum_{n=-N}^{N}c_n e^{in\omega t}|^2 & =\left(f(t)-\sum_{n=-N}^{N}c_n e^{in\omega t}\right)\left(f(t)-\sum_{n=-N}^{N}c_n e^{in\omega t}\right)^*                                       \\
                                               & =\left(f(t)-\sum_{n=-N}^{N}c_n e^{in\omega t}\right)\left(f^*(t)-\sum_{n=-N}^{N}c_n e^{-in\omega t}\right)                                      \\
                                               & =|f(t)|^2-\sum_{n=-N}^{N}(c_n^*f(t)e^{in\omega t}+c_n f^*(t)e^{-in\omega t})+\sum_{m,n=-N}^{N}c_m c_n^*e^{i(m-n)\omega t}
\end{align*}
将上式在一个周期上积分,我们知道
\[\int_{T}f(t)e^{in\omega t}\,dt=Tc_n,\int_{T}e^{i(m-n)\omega t}\,dt=\begin{cases}
        0 & \text{if }m\neq n \\
        T & \text{if }m=n
    \end{cases}\]
故\begin{align*}
      & \int_{T}|f(t)|^2\,dt-\sum_{n=-N}^{N}\left(c_n^*\int_{T}f(t)e^{in\omega t}\,dt+c_n \int_{T}f^*(t)e^{-in\omega t}\,dt\right)+\sum_{m,n=-N}^{N}c_m c_n^*\int_{T}e^{i(m-n)\omega t}\,dt \\
    = & \int_{T}|f(t)|^2\,dt-T\sum_{n=-N}^{N}(c_n^* c_n+c_n c_n^*)+T\sum_{n=-N}^{N}c_n^* c_n                                                                                     \\
    = & \int_{T}|f(t)|^2\,dt-T\sum_{n=-N}^{N}|c_n|^2
\end{align*}
这是非负函数的积分,积分值非负,即\begin{align*}
    \sum_{-\infty}^{\infty}|c_n|^2\leq \frac{1}{T}\int_{T}|f(t)|^2\,dt<\infty
\end{align*}

直接由狄利克雷条件证明逐点收敛性需要很专业的分析学工具,但我们可以适当地加强狄
利克雷条件,让$f(t)$\textbf{分段光滑} (piecewise smooth):
\[f\in PS([0,T])\Longleftrightarrow \text{除有限个点外f均可导,并且这些点是f的第一类间断点}\]
我们研究的多数函数是满足这样的性质的,并且我们将看到满足此条件会带来一些额外的
性质。此时就可以相对简单地证明逐点收敛性:
\[\lim_{N\to\infty}S_N^f(t_0)=\frac{f(t_0+)+f(t_0-)}{2}\]
\begin{align*}
    \intertext{\textbf{Proof:}}
    S_N^f(t_0)-\frac{f(t_0^+)+f(t_0^-)}{2} & =\frac{1}{T}\left(\int_{T}f(t_0-\tau)D_N(\tau)\,d\tau-\int_{0}^{\frac{T}{2}}f(t_0^+)D_N(\tau)\,d\tau-\int_{-\frac{T}{2}}^{0}f(t_0^-)D_N(\tau)\,d\tau\right) \\
                                                                     & =\frac{1}{T}\left(\int_{0}^{\frac{T}{2}}(f(t_0-\tau)-f(t_0^+))D_N(\tau)\,d\tau+\int_{-\frac{T}{2}}^{0}(f(t_0-\tau)-f(t_0^-))D_N(\tau)\,d\tau\right)         \\
    S_N^f(t_0)-\frac{f(t_0^+)+f(t_0^-)}{2}  & =\frac{1}{T}\int_{T}g(t)(e^{i(N+1)\omega t}-e^{iN\omega t})\,dt                                                                                  \\
    \text{其中}g(t)                                                    & :=\begin{cases}
                                                                             \frac{f(t_0+t)-f(t_0^-)}{e^{i\omega t}-1} & \text{if }-\frac{T}{2}<t_0<0 \\
                                                                             \frac{f(t_0+t)-f(t_0^+)}{e^{i\omega t}-1} & \text{if }0<t_0<\frac{T}{2}
                                                                         \end{cases}                                                      \\
    \text{由洛必达法则,$t\to 0$时,}\lim_{t\to 0^+}g(t)=                     & \lim_{t\to 0^+}\frac{f(t_0+t)-f(t_0^+)}{e^{i\omega t}}=\lim_{t\to 0^+}\frac{f'(t_0+t)}{ie^{i\omega t}}=\lim_{t\to 0^+}\frac{f'(t_0^+)}{i}
\end{align*}
$t\to 0^-$时同理。故g分段连续,当然是平方可积的,由贝塞尔不等式,
$g(t)$的傅里叶系数平方和收敛,通项趋于0,$S_N^f(t_0)-(f(t_0^+)+f(t_0^-))/2=C_{-(N+1)}-C_N\to 0$
,得证。

在分段光滑的条件下,容易得到$f'(t)$的傅里叶系数,注意微积分基本定理可以分区间使用:
\begin{equation}
    a'_n=n\omega b_n,b'_n=-n\omega a_n,c'_n=in\omega c_n
\end{equation}
以$c_n$为例:
\begin{align*}
    c'_n & =\frac{1}{T}\int_{T}f'(t)e^{-in\omega t}\,dt                                                         \\
         & =\frac{1}{T}\left.f(t)e^{-in\omega t}\right|_0^T+in\omega\int_{T}f(t)e^{-i\omega t}\,dt=in\omega c_n
\end{align*}

f的原函数F的傅里叶系数同理,并且只要\textbf{分段连续}(见\ref{sec:Fourier_Series})
即可保证f可积,但是我们必须保证F是周期函数,这要求f的直流分量为0:
\[F(t+T)-F(t)=\int_{T}f(t)dt=Tc_0=0,c_0=0\]
此时,用刚刚得到的公式(2.16)就可直接得到F的傅里叶系数:
\begin{equation}
    A_n=\frac{a_n}{n\omega},B_n=\frac{b_n}{n\omega},C_n=\frac{c_n}{in\omega}
\end{equation}

分段光滑还能够推出f的傅里叶级数\textbf{一致收敛}于f,从而可以逐项积分、逐项求
导。回顾数学分析中的魏尔斯特拉斯M判别法:对于函数项级数
$\sum_{n=1}^{\infty}f_n(x)$,如果存在正项级数$\sum_{n=1}^{\infty}M_n<\infty$
使得在区间E上$|f_n(x)|<M_n$,则$\sum_{n=1}^{\infty}f_n(x)$在E上绝对收敛且
一致收敛。对于上述命题,只需证明$\sum_{n=1}^{\infty}|c_n|<\infty$.直接应用
贝塞尔不等式是无效的,但可以通过一个小技巧完成证明:
\begin{align*}
    \text{记}c'_n\text{为f'的傅里叶系数},c'_n =in\omega c_n,                            &                                                                                \\
    \sum_{n=-\infty}^{\infty}|c_n|=   |c_0|+\sum_{n\neq 0}| \frac{c'_n}{n}|\leq & |c_0|+(\sum_{n\neq 0}\frac{1}{n^2})^{1/2}(\sum_{n\neq 0}|c'_n|^2)^{1/2}<\infty
\end{align*}
最后一步使用了柯西-施瓦兹不等式。

请读者思考:我们探究了指数形式傅里叶级数收敛的条件,对于三角函数形式的傅里叶级
数应该怎么办?

\end{document}