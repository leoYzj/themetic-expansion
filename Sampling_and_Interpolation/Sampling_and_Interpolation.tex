\documentclass{ctexart}

\usepackage{silence}
\usepackage[top=2cm, bottom=2cm, left=2.5cm, right=2.5cm]{geometry}
\usepackage{graphicx}
\usepackage{grffile}
\graphicspath{{D:/code_clone/Math_behind_Signal_and_System/Figures}}
\usepackage{amsmath,amssymb,amsfonts}
\usepackage{multicol}
\usepackage{varwidth}
\usepackage{dsfont}
\usepackage{hyperref}
\usepackage{fix-cm}
\usepackage{bm}
\usepackage{xcolor}
\usepackage{array}
\usepackage{booktabs}
\usepackage{float}
\usepackage{pifont}
\usepackage{enumitem}
\usepackage{tikz}
\usepackage{hyperref}
\usetikzlibrary{shapes,arrows,positioning,calc}
\usepackage{silence}
\usetikzlibrary{calc}
\ActivateWarningFilters
\WarningFilter{latex}{Font shape}
\WarningFilter{latex}{Some font shapes}
\vfuzz=100pt  % 垂直方向容忍度
\hfuzz=100pt  % 水平方向容忍度  
\vbadness=10000
\hbadness=10000
\overfullrule=0pt  % 不标记过满行
\allowdisplaybreaks
\raggedbottom

\newcommand{\shah}{\operatorname{III}}
% 便捷命令:在文中书写原函数在上下限的取值,例如 \evalat{F(x)}{a}{b} 输出为 \left.F(x)\right|_{a}^{b}
\newcommand{\evalat}[3]{\left.#1\right|_{#2}^{#3}}
\newcommand{\lr}[2]{%
    \begin{center}
    \begin{minipage}[t]{0.45\textwidth}
        \centering
        \allowdisplaybreaks
        \textcolor{blue}{%
            \begin{varwidth}{\linewidth}
            $\begin{aligned}
            #1
            \end{aligned}$
            \end{varwidth}
        }
    \end{minipage}
    \hfill
    \begin{minipage}[t]{0.45\textwidth}
        \centering
        \allowdisplaybreaks
        \textcolor{red}{%
            \begin{varwidth}{\linewidth}
            $\begin{aligned}
            #2
            \end{aligned}$
            \end{varwidth}
        }
    \end{minipage}
\end{center}
}
\newlist{circlist}{enumerate}{1}
\setlist[circlist]{
    label=\protect\ding{\numexpr171+\arabic*\relax},
    left=0pt
}

\begin{document}

\begin{center}
    \Huge\bfseries 取样与插值 
\end{center}

\section{狄拉克梳状分布}\label{sec:Sampling and Interpolation}

在研究周期现象时,经常会将一个函数周期化(periodize),例如,给定函数$f(t)$,要将
它变成周期为T的函数,一种标准的做法是“平移、相加”,即在不涉及收敛性的问题时,令
\[f_T(t)=\sum_{k=-\infty}^{\infty}f(t-kT)\]
在傅里叶变换的章节中,我们曾建立了$\delta_a$的平移性质:$\delta_a*f=\tau_a f$,
所以我们可以将以上周期化的操作抽象出来:
\[f_T(t)=\sum_{k=-\infty}^{\infty}f(t-kT)=\sum_{k=-\infty}^{\infty}\delta_{kT}*f(t)=\left(\sum_{k=-\infty}^{\infty}\delta_{kT}\right)*f(t)\]
将分布$\sum_{k=-\infty}^{\infty}\delta_{kT}$记为$\shah_T$,读作“shah”,在T=1
时可以将角标略去,直接写作$\shah$。由于其图像是等间距排列的一个个$\delta$,又称
之为\textbf{狄拉克梳状分布}(dirac comb).容易看出,这个分布具有两条基本性质,其一是前
面提到的周期化:$\shah_T*f(t)=f_T(t)$,其二是取样性质,它直接得自$\delta_a$的
取样性质:$\shah_T f(t)=\sum_{k=-\infty}^{\infty}f(kT)\delta_{kT}$。另外,
作为一个分布,有
\[\langle \shah_T,\varphi\rangle=\langle \sum_{k=-\infty}^{\infty}\delta_{kT},\varphi\rangle=\sum_{k=-\infty}^{\infty}\varphi(kT)\]
\begin{figure}[H]
    \centering
    \includegraphics[width=0.8\textwidth]{shah.jpeg}
\end{figure}
对于一个施瓦兹函数,以上的式子中均不会出现收敛性的问题。回顾施瓦兹函数类的定义:
\[\mathcal{S} =\{\varphi\in C^{\infty}(\mathbb{R} ):\lim_{|x|\to\infty}|x|^m\varphi^{(n)}(x)=0,\forall m,n\in\mathbb{N} \}\]
对任意的t,考察级数$\sum_{k=-\infty}^{\infty}f(t-kT)$的收敛性时,只需要将n取
为0,m取为不小于2的整数,则在$|k|$充分大时,级数比$C/n^2$更小,从而绝对收敛。注
意这种估计方式并不依赖于自变量t的选取,因此这个函数项级数一致收敛,其和函数会保留
$f(t)$的许多分析性质,例如我们可以逐项求导(从而和函数也无限阶可导)、逐项积分,
还可以交换求和符号与其他极限过程的顺序(见数学分析教材)。

时域上紧支(即在$|t|$充分大时函数恒为0)的函数称之为\textbf{时限函数},此时级数
$\sum_{k=-\infty}^{\infty}f(t-kT)$实际上是有限和,当然收敛。相应的,在频域上紧
支的函数称为\textbf{带限函数}。更为特殊的情况下,函数$f(t)$支于$[-T/2,T/2]$,
不论f定义在哪个域上,在相应的域上周期化所得的函数总是存在,并且有恒等式
\[f(x)=\Pi_{T}(f*\shah_T)(x)\]
这个看似直白的恒等式在傅里叶变换下能够带来一些新的东西,见本节后续内容。这里我们先建立
一个不需要$\shah$的傅里叶变换的结果。设带限函数$f(t)$满足$supp\mathcal{F} f\subset [-T/2,T/2]$,
则有恒等式$\mathcal{F} f=\Pi_T\mathcal{F} f$,对两边同时取傅里叶逆变换:\lr{
    f(t)&=\mathcal{F} ^{-1}\mathcal{F} f(t)\\
    &=\mathcal{F} ^{-1}(\Pi_T\mathcal{F} f)(t)\\
    &=\left(\mathcal{F} ^{-1}\Pi_T\right)*f(t)\\
    &=Tsinc(Tt)*f(t)
}{
    f(t)&=\mathcal{F} ^{-1}\mathcal{F} f(t)\\
    &=\mathcal{F} ^{-1}(\Pi_T\mathcal{F} f)(t)\\
    &=\frac{1}{2\pi}\left(\mathcal{F} ^{-1}\Pi_T\right)*f(t)\\
    &=\frac{T}{2\pi}Sa(Tt)*f(t)
}
我们发现对于带限函数,抽样信号能够起到类似$\delta$的卷积幺元的性质。

在建立分布$\shah$的傅里叶变换之前,我们先来讨论另外一个话题,它揭示了傅里叶级数与
傅里叶变换的深刻关系。设$suppf\subset [-T/2,T/2]$,
\lr{
    \mathcal{F} f(k)&=\int_{-\infty}^{\infty}f(t)e^{-2\pi ikt}\,dt\\
    &=\int_{-T/2}^{T/2}f(t)e^{-2\pi ikt}\,dt\\
    &=Tc_k
}{
    \mathcal{F} f(2\pi k)&=\int_{-\infty}^{\infty}f(t)e^{-2\pi ikt}\,dt\\
    &=\int_{-T/2}^{T/2}f(t)e^{-2\pi ikt}\,dt\\
    &=Tc_k
}
其中$c_k$是$f(t)$的周期化函数$f_T(t)$的傅里叶系数,也就是说,时限函数的傅里叶变
换在k(或$2\pi k$)处的值正比于其周期化函数的第k个傅里叶系数,这使得频域函数$\mathcal{F}f$
在整数处的值似乎蕴含了更多的信息。不仅如此,考虑更一般的函数$f\in\mathcal{S} $,记
其周期化函数为$\Phi=\shah*\varphi$,并将$\Phi$的傅里叶系数记为$\hat{\Phi}(k)$,
由$\Phi$的无限阶可导性可知其傅里叶级数绝对一致收敛于它本身,即
\[\Phi(t)=\sum_{k=-\infty}^{\infty}\hat{\Phi}(k)e^{2\pi ikt}\]
其中
\begin{align*}
    \hat{\Phi}(k)&=\int_{0}^{1}\Phi(t)e^{-2\pi ikt}\,dt\\
    &=\int_{0}^{1}\left(\sum_{n=-\infty}^{\infty}\varphi(t-n)\right)e^{-2\pi ikt}\,dt\\
    &=\sum_{n=-\infty}^{\infty}\left(\int_{0}^{1}\varphi(t-n)e^{-2\pi ikt}\,dt\right)\\
    &=\sum_{n=-\infty}^{\infty}\left(\int_{n}^{n+1}\varphi(t)e^{-2\pi ik(t+n)}\,dt\right)   &(t\to t+n)\\
    &=\sum_{n=-\infty}^{\infty}\int_{n}^{n+1}\varphi(t) e^{-2\pi ikt}\,dt\\
    &=\int_{-\infty}^{\infty}\varphi(t)e^{-2\pi ikt}\,dt
\end{align*}
即\lr{
    \hat{\Phi}(k)=\mathcal{F} \varphi(k)
}{
    \hat{\Phi}(k)=\mathcal{F} \varphi(2\pi k)
}
其中第二行到第三行的积分、求和换序用到了周期化函数作为函数项级数的一致收敛性。这个等式意味着对时
限函数的关系$\mathcal{F} f(k)=Tc_k$对施瓦兹函数都成立(以上是周期化为$T=1$的情况
,读者可以自行推广至周期为其他值的情况),进一步,在恒等式
\lr{
    \Phi(t)&=\sum_{k=-\infty}^{\infty}\hat{\Phi}(k)e^{2\pi ikt}\\
    &=\sum_{k=-\infty}^{\infty}\mathcal{F} \varphi(k)e^{2\pi ikt}
}{
    \Phi(t)&=\sum_{k=-\infty}^{\infty}\hat{\Phi}(k)e^{2\pi ikt}\\
    &=\sum_{k=-\infty}^{\infty}\mathcal{F} \varphi(2\pi k)e^{2\pi ikt}
}
中取$t=0$,则得到
\lr{\Phi(0)=\sum_{k=-\infty}^{\infty}\mathcal{F} \varphi(k)}{\Phi(0)=\sum_{k=-\infty}^{\infty}\mathcal{F} \varphi(2\pi k)}
又由\[\Phi=\shah*\varphi,\Phi(0)=\sum_{k=-\infty}^{\infty}\varphi(k)\]
得到\lr{
    \sum_{k=-\infty}^{\infty}\varphi(k)=\sum_{k=-\infty}^{\infty}\mathcal{F} \varphi(k)
}{
    \sum_{k=-\infty}^{\infty}\varphi(k)=\sum_{k=-\infty}^{\infty}\mathcal{F} \varphi(2\pi k)
}
这个结果称为\textbf{泊松求和公式}(the Poisson summation formula)。我们当然也可以
对$\shah_T*\varphi$做类似的推导,但其结论远不如以上两式简洁,这实际上是整数集$\mathbb{Z}$
的自对偶性的体现。

让我们回到$\shah$分布的傅里叶变换的话题,可以借助泊松求和公式得到:
\lr{
    \langle \mathcal{F} \shah,\varphi \rangle&=\langle \shah,\mathcal{F} \varphi\rangle\\
    &=\sum_{k=-\infty}^{\infty}\mathcal{F} \varphi(k)\\
    &=\sum_{k=-\infty}^{\infty}\varphi(k)\\
    &=\langle \shah,\varphi\rangle
}{
    \langle \mathcal{F} \shah_{2\pi},\varphi \rangle&=\langle \shah_{2\pi},\mathcal{F} \varphi\rangle\\
    &=\sum_{k=-\infty}^{\infty}\mathcal{F} \varphi(2\pi k)\\
    &=\sum_{k=-\infty}^{\infty}\varphi(k)\\
    &=\langle \shah,\varphi\rangle
}
即\lr{
    \mathcal{F} \shah=\shah
}{
    \mathcal{F} \shah_{2\pi}=\shah
}
可以看到,频率形式下$\shah$分布的傅里叶变换具有自对偶性,即$\shah\overset{\mathcal{F} }{\longleftrightarrow}\shah$,
而角频率形式下需要对$\shah$分布做伸缩变换,这启发我们讨论$\shah$以及更基本的$\delta$
在伸缩变换下的关系。可以将$\delta$作为形式上的函数进行推导,但这里采用更直接的分布
伸缩的定义:($a>0$)
\begin{align*}
    \langle \sigma_a\delta,\varphi\rangle&=\langle \delta,\frac{1}{a}\sigma_{1/a}\varphi\rangle\\
    &=\frac{1}{a}\varphi(0)=\langle \frac{1}{a}\delta,\varphi\rangle\\
    &\Rightarrow \sigma_a\delta=\frac{1}{a}\delta
\end{align*}
$a<0$的情况可直接从$\delta$是偶分布得出。对于$\shah$分布的伸缩,也可同理得到:\begin{align*}
    \langle \sigma_a\shah,\varphi\rangle&=\langle \shah,\frac{1}{a}\sigma_{1/a}\varphi\rangle\\
    &=\frac{1}{a}\sum_{k=-\infty}^{\infty}\varphi(ax-k)\\
    &=\frac{1}{a}\langle \shah_{1/a},\varphi\rangle\\
    &\Rightarrow \sigma_a\shah=\frac{1}{a}\shah_{1/a},\shah_T=\frac{1}{T}\sigma_{1/T}\shah
\end{align*}

值得注意的是,不论从直观上还是从形式计算上,$\shah$分布似乎有另一个傅里叶变换:
\lr{
    \mathcal{F} \shah&=\sum_{k=-\infty}^{\infty}\mathcal{F} \delta_k\\
    &=\sum_{k=-\infty}^{\infty}e^{-2\pi ikt}
}{
    \mathcal{F} \shah_{2\pi}&=\sum_{k=-\infty}^{\infty}\mathcal{F} \delta_{2k\pi}\\
    &=\sum_{k=-\infty}^{\infty}e^{-2\pi ikt}
}
实际上,这个结果是正确的,并且可以由此得到$\shah$分布作为形式上的函数的另一种定义:
\begin{equation}
    \shah(x)=\sum_{k=-\infty}^{\infty}e^{-2\pi ikx}
\end{equation}
右边正是傅里叶级数的章节中提到的狄利克雷核在$N\to\infty$时的
式子,换言之函数项级数$\sum_{k=-\infty}^{\infty}e^{-2\pi ikx}$逐点地趋于$\shah$,
我们不仅发现了它在多数区间上都趋于0,还发现在$x=n(n\in\mathbb{Z})$处全为1的“级数”
似乎以$\delta$的速度趋于无穷。实际上我们只能得到$\mathbb{R}-\mathbb{Z}$上的收敛
性,整数处的分布只能在弱收敛的意义下定义,即将函数
项级数$\sum_{k=-\infty}^{\infty}e^{-2\pi ikx}$的每一项视为分布,它们作用在任意
测试函数$\varphi$上的结果在求和项数充分大时相当于$\delta$分布作用于$\varphi$的结
果。

根据角频率形式下的$\shah$分布的傅里叶变换,可以想到$\shah$的伸缩的傅里叶变换仍具有
$\shah$的形式,下面就来找出具体的公式:\lr{
    \mathcal{F} \shah_T&=\mathcal{F} [\frac{1}{T}\sigma_{1/T}\shah]\\
    &=\frac{1}{T}\cdot T\sigma_T\mathcal{F} \shah\\
    &=\sigma_T\shah=\frac{1}{T}\shah_{1/T}
}{
    \mathcal{F} \shah_T&=\mathcal{F} [\frac{2\pi}{T}\sigma_{2\pi/T}\shah_{2\pi}]\\
    &=\frac{2\pi}{T}\cdot \frac{T}{2\pi}\sigma_{T/2\pi}\mathcal{F} \shah_{2\pi}\\
    &=\sigma_{T/2\pi}\shah=\frac{2\pi}{T}\shah_{2\pi/T}
}
即\lr{
    \mathcal{F} \shah_T=\frac{1}{T}\shah_{1/T}
}{
    \mathcal{F} \shah_T=\frac{2\pi}{T}\shah_{2\pi/T}
}
可以看到,傅里叶变换前后$\delta$的位置具有倒数关系,以频率为自变量的傅里叶变换,
$T\cdot\frac{1}{T}=1$,以角频率为自变量做傅里叶变换,$T\cdot\frac{2\pi}{T}=2\pi$,
其联系还是在于$\omega=2\pi\xi$。事实上,在多维傅里叶变换的理论中,我们还将看到更为
深刻的关系:$\shah$分布的$\delta$的位置称为“格”,而其傅里叶变换的$\delta$的位置为
其“对偶格”,特别地,一维的“对偶格”就是以上所述的“倒数”关系。直观上,我们曾提出对函
数和分布的一维傅里叶变换的伸缩定理:
\[f(at)\overset{\mathcal{F} }{\longleftrightarrow}\frac{1}{|a|}F(\frac{\xi}{a}),\sigma_a T\overset{\mathcal{F} }{\longleftrightarrow}\frac{1}{|a|}\sigma_{1/a}\mathcal{F} T\]
它同样揭示了时域伸而频域缩,时域缩而频域伸的关系,而$\shah_T$为我们提供了一种新的
视角。二维$\shah$分布在
$\mathbb{R}^3$中的图像就像一个“钉床”,因此又可以将$\shah$分布称为\textbf{钉床函数}(bed of nails)。

讨论完$\shah$的傅里叶变换,就可以回到一开始的话题:对于时限函数$f(t),suppf\subset[-T/2,T/2]$,研究恒等式
\[f(x)=\Pi_{T}(f*\shah_T)(x)\]
通过傅里叶变换告诉我们的内容。对$f*\shah_T$做傅里叶变换,有
\lr{
    \mathcal{F} (f*\shah_T)(\xi)&=\mathcal{F} f(\xi)\cdot\mathcal{F} \shah_T(\xi)\\
    &=\mathcal{F} f(\xi)\cdot\frac{1}{T}\shah_{1/T}(\xi)\\
    &=\frac{1}{T}\sum_{k=-\infty}^{\infty}\mathcal{F} f(k/T)\delta_{k/T}(\xi)
}{
    \mathcal{F} (f*\shah_T)(\omega)&=\mathcal{F} f(\omega)\cdot\mathcal{F} \shah_T(\omega)\\
    &=\mathcal{F} f(\omega)\cdot\frac{2\pi}{T}\shah_{2\pi/T}(\omega)\\
    &=\frac{2\pi}{T}\sum_{k=-\infty}^{\infty}\mathcal{F} f(2\pi k/T)\delta_{2\pi k/T}(\omega)
}
其物理意义是,将f以周期T进行周期化时,其频谱会剩下频率为$k/T$处,也即角频率为$2\pi k/T$
处的值,并且在这些频率处以$\delta$分布的形式“趋于无穷”。

对恒等式两边同时做傅里叶变换,有\lr{
    \mathcal{F} f(\xi)&=\mathcal{F} [\Pi_T(f*\shah_T)](\xi)\\
    &=\mathcal{F} [\Pi_T](\xi)\mathcal{F} (f*\shah_T)(\xi)\\
    &=Tsinc(T\xi)*\left(\frac{1}{T}\sum_{k=-\infty}^{\infty}\mathcal{F} f(k/T)\delta_{k/T}(\xi)\right)\\
    &=\sum_{k=-\infty}^{\infty}\mathcal{F} f(k/T)sinc\left(T(\xi-\frac{k}{T})\right)\\
    &=\sum_{k=-\infty}^{\infty}\mathcal{F} f(k/T)sinc(T\xi-k)
}{
    \mathcal{F} f(\omega)&=\mathcal{F} [\Pi_T(f*\shah_T)](\omega)\\
    &=\mathcal{F} [\Pi_T](\omega)\mathcal{F} (f*\shah_T)(\omega)\\
    &=2\pi TSa(T\omega)*\left(\frac{2\pi}{T}\sum_{k=-\infty}^{\infty}\mathcal{F} f(2\pi k/T)\delta_{2\pi k/T}(\omega)\right)\\
    &=4\pi^2\sum_{k=-\infty}^{\infty}\mathcal{F} f(2\pi k/T)Sa\left(T(\omega-\frac{2\pi k}{T})\right)\\
    &=4\pi^2\sum_{k=-\infty}^{\infty}\mathcal{F} f(2\pi k/T)Sa(T\omega-2\pi k)
}
可以看到,我们用$\mathcal{F} f(k/T)sinc(T\xi-k)$或$\mathcal{F} f(2\pi k/T)Sa(T\omega-2\pi k)$
表示出了函数$\mathcal{F} f$,而表达式中的抽样函数完全依赖于所进行的周期化的周期T,也
就是说我们只需要用$\mathcal{F} f(k/T)$或$\mathcal{F} f(2\pi k/T)$处的值,就能还原
处完整的$\mathcal{F} f$,进一步,对于带限函数$f(t),supp\mathcal{F} f\subset[-\nu_m,\nu_m]$,可以提出
完全类似的恒等式,其中$T=1/2\nu_m$,对应频率$f_N=1/T$被称为\textbf{奈奎斯特频率}(Nyquist frequency):
\[\mathcal{F} f(x)=\Pi_T(\mathcal{F} f*\shah_T)(x)\]
这个式子称为\textbf{抽样定理}(the sampling thoerem)。对其做傅里叶反变换,有\lr{
    f(t)&=\mathcal{F} ^{-1} [\Pi_T(\mathcal{F} f*\shah_T)](t)\\
    &=T sinc(Tt)*\left(\frac{1}{T}\sum_{k=-\infty}^{\infty}f(k)\delta_k(t)\right)\\
    &=\sum_{k=-\infty}^{\infty}f(k/T)sinc\left(T(t-\frac{k}{T})\right)\\
    &=\sum_{k=-\infty}^{\infty}f(k)sinc(Tt-k)
}{
    f(t)&=\mathcal{F} ^{-1} [\Pi_T(\mathcal{F} f*\shah_T)](t)\\
    &=2\pi TSa(Tt)*\left(\frac{2\pi}{T}\sum_{k=-\infty}^{\infty}f(k)\delta_{2\pi k}(t)\right)\\
    &=4\pi^2\sum_{k=-\infty}^{\infty}f(2\pi k/T)Sa\left(T(t-\frac{2\pi k}{T})\right)\\
    &=4\pi^2\sum_{k=-\infty}^{\infty}f(k)Sa(Tt-2\pi k)
}
这正是\textbf{理想抽样}下,由抽样函数恢复出原函数的方法。至此可以看到使用角频率做傅里
叶变换将使理论推导极不方便、容易出错,因此在讨论取样与插值问题的\ref{sec:Sampling and Interpolation}
和\ref{sec:advanced_Interpolation}中将一直使用频率做傅里叶变换。利用$\shah$分布的
抽样性质进行抽样,就好比将函数转化为一组无穷基底$\delta_k(k\in\mathbb{Z})$上的向量:
\[f(t)\shah(t)=\sum_{k=-\infty}^{\infty}f(k)\delta_k\text{相当于}x[k]=f[k]\text{或}\left(\cdots,f(-2),f(-1),f(0),f(1),f(2),\cdots\right)^T\]
这一过程称为理想抽样。实际问题中我们不可能获得$\delta$分布和$\shah$分布,也不可能进
行无限多次抽样,解决这个问题的一种方式是做近似,另一种方式是有限采样,将在下一节讨论。

我们来明确一下取样与插值要讨论的问题:\begin{itemize}
    \item 设有确定信号$f(t)$,通过取样的方式找出它在若干点处的\textbf{样值}$f(t_0),f(t_1),f(t_2),\cdots$,在怎样的情况下可以借助这组值还原出$f(t)$,即在离散的数据点之间应如何“插入”新的数据点使之成为一条连续的曲线(这也是名称“插值”的由来)
\end{itemize}
一般地,相邻取样点的间隔一定,记为$T_s$,称为\textbf{抽样间隔},将$f_s=1/T_s$称为
\textbf{抽样频率},将$\omega_s=2\pi f_s=2\pi/T_s$称为\textbf{抽样角频率}。类似于用
像素表示图片,利用抽样所得的离散信号$f_d[n]=f(nT_s)$表示信息,相当于用离散信号表示连
续信号,能够大大降低传输和存储的成本。一个数字处理系统或数字传输系统的第一个环节就是对
信号进行抽样,随后进行量化编码、数字传输或处理,到信号接收方或需要使用信号时再进行信号
恢复,这是取样与插值问题的一种引出方式。

我们刚才由抽样信号恢复出原信号的过程,实际上蕴含着\textbf{香农采样定理}(Shannon sampling theorem)
的思想。香农采样定理的内容是,如果已知信号的频谱支于$[-\nu_m,\nu_m]$,则采样频率
必须大于$2\nu_m$,即奈奎斯特频率$f_N$,才能还原出原信号。可以看出,奈奎斯特频率正是信
号的\textbf{带宽}(bandwidth)。

直观上,采样频率越高,所需要的成本就越高,提供的信息就越多,而如果采样频率低于$f_N$,
将发生\textbf{混叠}(alias)现象,还原所得的信号将产生失真。例如一个余弦信号,较低的采
样频率可能使我们的样本点全部位于余弦信号的零点,此时显然无法还原余弦信号。即便样本点不
在零点,也可能还原出其他的信号:
\begin{figure}[H]
    \centering
    \includegraphics[width=0.4\textwidth]{cos.jpeg}
\end{figure}
这正是混叠现象。从频域上不难理解它,在建立抽样定理时我们曾要求
$f(t),supp\mathcal{F} f\subset[-\nu_m,\nu_m]$,利用由此得到的公式
\lr{
    f(t)=\sum_{k=-\infty}^{\infty}f(k)sinc(Tt-k)
}{
    f(t)=4\pi^2\sum_{k=-\infty}^{\infty}f(k)Sa(Tt-2\pi k)
}
还原函数当然是有误差的,因为在将频域函数周期化时,两个平移后的频域函数会有一部分混在一
起,再使用$\Pi$函数进行加窗时,会丢掉一部分原有频谱,并叠加一部分平移后的频谱,如图所示。
\begin{figure}[H]
    \centering
    \includegraphics[width=0.8\textwidth]{alias.jpeg}
\end{figure}

\section{取样与插值进阶}\label{sec:advanced_Interpolation}

插值的雏形来自于古代天文学,最简单的插值就是通过观察星体在几个时刻的位置推测中间时刻的
位置。历史上,最早对插值问题进行系统研究的是牛顿和拉格朗日。牛顿提出的牛顿插值公式是组
合数学讨论的话题,他通过引入离散信号的差商的概念,提出了一种使用三角形差商表构造过若干
指定数据点的多项式的方法;相比之下拉格朗日的拉格朗日插值法要简单得多,其想法的关键在于
构造在特定数据点处非0,而在另外所有数据点处均为0的多项式:对于一组互异的数据点
 \[(x_0,y_0),(x_1,y_1),\dots,(x_n,y_n)\]其拉格朗日插值多项式为
\[
L(x) = \sum_{i=0}^{n} y_i \ell_i(x)
\]
其中 $\ell_i(x)$ 是拉格朗日基函数,满足$\ell_i(x_i)=1,\ell_i(x_j)=0(j\neq i)$:
\[
\ell_i(x) = \prod_{\substack{j=0 \\ j \neq i}}^{n} \frac{x - x_j}{x_i - x_j}
\]
该多项式满足 $L(x_i) = y_i,i=0,1,\dots,n$。这个方法后来被埃尔米特改进,
不仅能保证函数值相等,还能保证若干阶导数的值相等。

然而,利用多项式进行插值对原信号的逼近效果十分有限,一些数学家们开始转向研究如何重建整
个信号。1948年,克劳德·香农发表了他的巨著《通信的数学原理》,明确提出了他的香农采样定
理,并给出了上一节中还原带限信号的公式。这个定理在我们上一节中所进行的插值流程下是正确
的,但要还原大带宽信号需要很高的抽样频率,会对硬件设备提出很高的要求。我们可以通过另外
的插值流程还原信号,例如对于一些信号可以利用其稀疏性和相关性,用远低于$f_N$的采样频率
还原信号。这一节就来讨论真实系统中的插值方法,包括自然采样、零阶抽样保持、一阶抽样保持
以及降低采样率的一些办法。最后,还将给出有限采样公式,它从数学上给出了用有限的采样点还
原带限周期信号的方法。

\end{document}