\documentclass{ctexart}

\usepackage{silence}
\usepackage[top=2cm, bottom=2cm, left=2.5cm, right=2.5cm]{geometry}
\usepackage{graphicx}
\usepackage{grffile}
\graphicspath{{D:/code/Math_behind_Signal_and_System/Figures}}
\usepackage{amsmath,amssymb,amsfonts}
\usepackage{multicol}
\usepackage{varwidth}
\usepackage{dsfont}
\usepackage{hyperref}
\usepackage{fix-cm}
\usepackage{bm}
\usepackage{xcolor}
\usepackage{array}
\usepackage{booktabs}
\usepackage{float}
\usepackage{pifont}
\usepackage{enumitem}
\usepackage{tikz}
\usepackage{hyperref}
\usetikzlibrary{shapes,arrows,positioning,calc}
\usepackage{silence}
\usetikzlibrary{calc}
\ActivateWarningFilters
\WarningFilter{latex}{Font shape}
\WarningFilter{latex}{Some font shapes}
\vfuzz=100pt  % 垂直方向容忍度
\hfuzz=100pt  % 水平方向容忍度  
\vbadness=10000
\hbadness=10000
\overfullrule=0pt  % 不标记过满行
\allowdisplaybreaks
\raggedbottom

\newcommand{\shah}{\operatorname{III}}
% 便捷命令:在文中书写原函数在上下限的取值,例如 \evalat{F(x)}{a}{b} 输出为 \left.F(x)\right|_{a}^{b}
\newcommand{\evalat}[3]{\left.#1\right|_{#2}^{#3}}
\newcommand{\lr}[2]{%
    \begin{center}
    \begin{minipage}[t]{0.45\textwidth}
        \centering
        \allowdisplaybreaks
        \textcolor{blue}{%
            \begin{varwidth}{\linewidth}
            $\begin{aligned}
            #1
            \end{aligned}$
            \end{varwidth}
        }
    \end{minipage}
    \hfill
    \begin{minipage}[t]{0.45\textwidth}
        \centering
        \allowdisplaybreaks
        \textcolor{red}{%
            \begin{varwidth}{\linewidth}
            $\begin{aligned}
            #2
            \end{aligned}$
            \end{varwidth}
        }
    \end{minipage}
\end{center}
}
\newlist{circlist}{enumerate}{1}
\setlist[circlist]{
    label=\protect\ding{\numexpr171+\arabic*\relax},
    left=0pt
}

\begin{document}

\section{系统概述}\label{sec:System}
对于单输入单输出系统,将输入信号称为激励 (excitation),输出信号称为响应 (response)
,并将时域信号分别用$e(t),r(t)$表示,如果系统用字母H表示,可以记$r(t)=H[e(t)]$
,或更简洁地,$r(t)=He(t)$.H是函数空间上的映射。如果H是线性的,即
\[\forall a,b\in\mathbb{C},\forall e_1(t),e_2(t),H[ae_1(t)+be_2(t)]=aH[e_1(t)]+bH[e_2(t)]\]
就说系统是\textbf{线性系统}(linear system),且满足叠加法则 (principle of superposition)。
有时将线性系统记为L.

一个系统在e(t)=0时,也可能有响应,这样的响应称为\textbf{零输入响应}
(zero input response,$r_{z.i}(t)$)。相应地,r(t)在没有激励时为0,则输入
e(t)产生的响应称为\textbf{零状态响应}(zero state response,$r_{z.s}(t)$),对
于线性系统显然零输入响应为0,此时我们定义零状态响应有线性性的系统为线性系统。
\textbf{全响应}$r(t)=r_{z.i}(t)+r_{z.s}(t)$.

当$e(t)=\delta(t)$时,$r_{z.s}(t)$为\textbf{单位脉冲响应},在不涉及零输入
响应时就说r(t)为单位脉冲响应。当$e(t)=u(t)$时,响应r(t)称为\textbf{单位阶跃响应}。
习惯上,将单位脉冲响应记为h(t),单位阶跃响应记为g(t).

系统还有一些其他的特性,下面一一进行说明。\\
1.\textbf{时不变性}:表示一个系统的输出不依赖于输入信号施加于系统的时间,输
入信号发生时移,输出信号也发生相同的时移,即$\forall b\in\mathbb{R},H[e(t-b)]=r(t-b)$,
用第二章中定义的时移算子的符号,时不变性可以记为$H\tau_b=\tau_b H$,换言之,时
移和经过系统可交换,这与我们的直观相符。
\textbf{线性时不变系统}(Linear Time-invariant System,LTI)满足$r(t)=(h*e)(t)$
,h为单位脉冲响应,将在后文介绍。在考虑系统的初始状态时,只要零状态响应具有线性
和时不变性,就说系统是线性时不变系统。
\\
\noindent 2.\textbf{因果性}:表示一个系统在有激励时,才会出现响应,或者说
$r(t_0)$仅依赖于e(t)在$t<t_0$时的值(这里不等号不取等是标准的定义方式,我
们马上会看到它的作用),即
\begin{equation}
    \forall t_0,(e_1(t)=e_2(t)\ for\ t<t_0)\Rightarrow (r_1(t)=r_2(t)\ for\ t<t_0)\label{eq:4.1}
\end{equation}
这个条件称为\textbf{因果性条件}(casuality condition)。对于线性系统,容易看
出其因果性条件等价于
\begin{equation}
    \forall t_0,(e(t)=0\ for\ t<t_0)\Rightarrow (r(t)=0\ for\ t<t_0)\label{eq:4.2}
\end{equation}
如果系统还有时不变性,则因果性条件等价于
\begin{equation}
    (e(t)=0\ for\ t<0)\Rightarrow(r(t)=0\ for\ t<0)\label{eq:4.3}
\end{equation}
或者更简便的
\begin{equation}
    h(t)=0\ for\ t<0\label{eq:4.4}
\end{equation}
其中h为单位脉冲响应。只要取$t_0=0$,就从因果性条件推出条件\ref{eq:4.3};用
时不变性,条件\ref{eq:4.3}可以推出条件\ref{eq:4.2}。对于单位脉冲响应,从
$\delta=0\ for\ t<0$可以推出$h(t)=0\ for\ t<0$;如果$h(t)=0,e(t)=0\ for\ t<0$
,则根据卷积的章节中的结果,有$r(t)=(h*e)(t)=0\ for\ t<0$.

\noindent 3.\textbf{稳定性}:表示一个系统在激励信号有界时,响应也是有界的,
即bounded-input bounded-output(BIBO)。工程上,一个实用系统在所有可能条件下
都保持稳定时至关重要的。

\noindent 4.\textbf{记忆性}:表示一个系统在$t_0$时刻的响应不仅与该时刻的输
入有关,还与其他时刻的输入有关,此时称系统是\textbf{记忆系统/动态系统},与之
相对的是\textbf{即时系统},在$t_0$时刻的响应仅与$t_0$时刻的激励有关。

\noindent 5.\textbf{可逆性}:H(作为映射)如果是单射,则也是双射(因为我们
不关注其值域),于是H是可逆的。换句话说,一种响应仅可能对应唯一的激励。

以下列举一些常见的线性系统。

\noindent 例1.1.时域乘积:\begin{equation}
    r(t)=f(t)e(t)
\end{equation}
例如描述开关开闭的系统,$f(t)=\Pi_T(t)$或$u(t)$,描述取样的系统,$f(t)=\shah(t)$.

\noindent 例1.2.矩阵乘法:对于n维的离散激励信号$\mathbf{v}$,\begin{equation}
    \mathbf{w}=A\mathbf{v},A\in M_{n\times n}(\mathbb{C})
\end{equation}
例如,线性动力系统中,初值问题$\mathbf{\dot{x}}(t)=A\mathbf{x}(t)$的解
为$H[\mathbf{v}]=e^{At}\mathbf{v}$.
线性代数中,如果一个线性算子L的在某一组基下的矩阵表示是A,则将L的转置定义为用
$A^T$表示的线性算子,其中T表示矩阵的转置,如果矩阵A是对称的,则算子L称为对称
算子;L的共轭转置定义为用$A^H$表示的算子,其中H表示矩阵的共轭转置,如果A是埃尔
米特的(Hermitian),即$A^H=A$,则称算子L是埃尔米特的,并说L是\textbf{自伴算子}
(self-adjoint operator).对于矩阵乘法系统,可以根据A的性质定义
\textbf{对称系统、自伴系统}。

\noindent 例1.3.积分:\begin{equation}
    r(t)=\int_{a}^{b}k(x,y)e(y)\,dy
\end{equation}
其中,$k(x,y)$称为核函数(kernel),$\int_{a}^{b}k(x,y)e(y)\,dy$称为对核积分
(integrating against a kernel).对核积分非常类似于连续版本的矩阵乘积,我们同
样可以定义:\begin{itemize}
    \item 上述系统的转置系统描述为$e(t)\mapsto\int_{a}^{b}k(y,x)e(y)\,dy$
    \item 如果$k(x,y)=k(y,x)$,称k是对称的并将系统称为\textbf{对称系统}
    \item 如果$k(x,y)=\overline{k(y,x)}$,称k是埃尔米特的并将系统称为\textbf{自伴系统}。
\end{itemize}

\noindent 例1.4.卷积

取定连续信号$g,r(t)=g*e(t)$;取定周期离散信号$\mathbf{h},r[n]=\mathbf{h*e}[n]$.
在标准正交基底下,卷积系统的矩阵是一个循环矩阵(circulant matrix)或托普利兹矩
阵(Toepliz matrix):\[\begin{pmatrix}
    h[0] & h[N-1] & \cdots & h[1]\\
    h[1] & h[0] & \cdots &  h[2]\\
    \vdots & \vdots & \ddots & \vdots\\
    h[N-1] & h[N-2] & \cdots & h[0]
\end{pmatrix}\]

\noindent 例1.5.信号平移:取定时延b,$r(t)=e(t-b)$;取定整数m,$r[n]=e[n-m]$.

\noindent 例1.6.傅里叶变换:$r(t)=\mathcal{F} e(t)$.在一些光学仪器中可以实
现空间上的傅里叶变换,这是变量t是多维的空间变量,例如矩孔夫琅禾费衍射装置。作
为径向对称函数的傅里叶变换的特例,零阶汉克尔变换可以用圆孔夫琅禾费衍射装置实现。

下面介绍系统的\textbf{级联}(cascade)。设有两线性系统K,L,将两系统级联之后,构
成的新系统仍构成线性系统:\begin{align*}
    LK[ae_1+be_2]=L[aK[e_1]+bK[e_2]]=aLK[e_1]+bLK[e_2]
\end{align*}
对于周期离散信号,两个系统都是矩阵乘法,则级联系统的矩阵是两矩阵的乘积,这
正是定义矩阵乘法的原因之一;类似地,对于连续信号,如果两个系统都是对核积分系统:
\[K[e(t)](x)=\int_{a}^{b}k(x,y)e(y)\,dy,L[e(t)](x)=\int_{a}^{b}l(x,y)e(y)\,dy\]
则级联系统$H=LK$也是对核积分系统,并且核函数为\begin{equation}
    h(x,z)=L_y(k(y,z))(x)=\int_{a}^{b}l(x,y)k(y,z)\,dy
\end{equation}
这里$L_y$表示$k(x,y)$作为y的函数输入系统L,输出为x,y的双变量函数。\\
\textbf{Proof:}\begin{align*}
    H[e(z)](x)&=LK[e(z)](x)\\
    &=\int_{a}^{b}l(x,y)\left(\int_{a}^{b}k(y,z)e(z)\,dz\right)\,dy\\
    &=\int_{a}^{b}e(z)\,dz\int_{a}^{b}l(x,y)k(y,z)\,dy\\
    &=\int_{a}^{b}h(x,z)e(z)\,dz
\end{align*}
可以看到,只要我们使用的函数满足富比尼定理(Fubini's thoerem)的条件,就可以交
换积分次序,并得到前文中断言的证明。

线性系统H的\textbf{脉冲响应}定义为$h(x,y)=H_x[\delta(x-y)]$,这与前文中定
义的单变量函数$h(t)$略有区别。信号与
$\delta$的卷积仍为它本身:
\[f(t)=(f*\delta)(t)=\int_{-\infty}^{\infty}f(x)\delta(t-x)\,dx\]
因此可以说,信号先经过了一个与$\delta$卷积的系统才进入系统H。配合前文中关于级
联系统的结果,如果系统H是对核的无穷限积分,就有\textbf{叠加定理}(superposition thoerem)
或称\textbf{施瓦兹核定理}(Schwartz kernel thoerem):\begin{equation}
    r(x)=\int_{-\infty}^{\infty}h(x,y)e(y)\,dy
\end{equation}
\textbf{Proof:}\begin{align*}
    H[e(y)]&=H[e*\delta(y)]\\
    &=H[\int_{-\infty}^{\infty}\delta(x-y)e(y)\,dy]\\
    &=\int_{-\infty}^{\infty}H[\delta(x-y)]e(y)\,dy\\
    &=\int_{-\infty}^{\infty}h(x,y)e(y)\,dy
\end{align*}
反过来,如果一个系统是激励信号与脉冲响应的含参积分,那么这个系统也是线性的,
这得自积分的线性性。

对于时不变系统,这个定理具有更加优美的形式。沿用前面的定义,时不变系统的脉冲响应为
\[h(x,y)=H_x[\delta(x-y)]=H[\tau_y\delta(x)]=\tau_y H[\delta(x)]\]
按照单变量的脉冲响应的定义,这就是$\tau_y h(x)=h(x-y)$,它只依赖于x、y的差值,
带入叠加定理:\begin{align*}
    r(x)&=\int_{-\infty}^{\infty}h(x-y)e(y)\,dy\\
    &=(h*e)(t)
\end{align*}
也就是说,信号经过线性时不变系统相当于与这个系统的单位脉冲响应做卷积。下面来验证,
如果一个系统是激励信号与脉冲响应的卷积,那么这个系统也是线性时不变系统。线性性是
显然的,我们来验证$H\tau_b=\tau_b H$:\begin{align*}
    H[\tau_b e(y)](x)=(h*\tau_b e)(x)=\tau_b(h*e)(x)=\tau_b H[e(y)]
\end{align*}
总之,“一个系统是线性系统”等价于“响应是激励对脉冲响应积分”,而“是线性时不变系统”
等价于“响应是激励与脉冲响应的卷积”。

看到卷积的结构,当然会想到对$r(t)=(h*e)(t)$做傅里叶变换,用对应的大写字母表示,
即$R(\xi)=H(\xi)E(\xi)$或$R(\omega)=H(\omega)E(\omega)$,也就是说,在频域上
线性时不变系统就是乘以脉冲响应的频域形式。

\noindent 例1.1.求线性时不变系统的单位阶跃响应$g(t)$:\begin{align*}
    g(t)&=H[u(t)]=(u*h)(t)\\
    &=\int_{-\infty}^{\infty}u(t-y)h(y)\,dy\\
    &=\int_{-\infty}^{t}h(y)\,dy
\end{align*}
可以看到,单位阶跃响应就是单位脉冲响应的积分,利用这个性质,可以快速地求出单位阶
跃响应,在使用卷积来描述线性时不变系统时,这个性质很容易得到,但在一些其他的线性
时不变系统中很难想到具有这样的性质,例如下一节将讨论的系统。

\section{微分方程}\label{sec:ODE}

除了上一节中提到的几种系统,还有一种典型的具有因果性的线性时不变系统是用常系数线
性微分方程表示的,在学习数学分析时我们曾遇到过这种微分方程,下面先回忆在数学分析
中对于这种方程的处理方法。一个形如
\[a_n y^{(n)}+a_{n-1}y^{(n-1)}+\cdot +a_1 y'+a_0 y=f\]
称为\textbf{n阶常系数线性微分方程},等式左侧的y和f都是t的函数,特别地,如果
$f(t)=0$,称之为\textbf{n阶常系数线性齐次微分方程}。不难看出,如果找到了n阶常
系数线性齐次微分方程的两个解$y_1,y_2$,则$ay_1+by_2$也是方程的解;另外,如果带
入$y=e^{\lambda t}$并消去这一项,则微分方程化为其\textbf{特征方程}:
\[a_n\lambda^n+a_{n-1}\lambda^{n-1}+\cdots+a_1\lambda+a_0=0\]
根据代数基本定理,$\lambda$在复数域$\mathbb{C}$中有n个解(称之为\textbf{特征根})
,即微分方程一定有n个形如$e^{\lambda t}$的解。因此,对于常系数线性微分方程,我
们的标准处理方法是:
\begin{enumerate}
    \item 先令$f(t)=0$,得到其对应的\textbf{齐次方程},解其方程的特征方程得到n个\textbf{齐次解}(homogeneous solution),记为$y_h(t)$;
    \item 找到一个\textbf{特解}(particular solution)使得它恰好满足原方程,记为$y_p(t)$;
    \item 将齐次解的线性组合与特解相加,得到\textbf{完全解}:$y(t)=y_h(t)+y_p(t)$。
\end{enumerate}
对于特解,需要一定的配凑技巧,特别地,如果f为一个t的多项式与某个齐次解的乘积,则
需要带入固定形式的特解求解其中的参数(以下默认P、Q为多项式):\begin{itemize}
    \item 如果$f(t)=c$,c为常数,则特解$y_p(t)$也为常数;
    \item 如果$f(t)=e^{\beta t}$,$\beta$不是特征根,则特解$y_p(t)=ce^{\beta t}$,c为常数;
    \item 如果$f(t)$为关于t的n次多项式,则特解$y_p(t)$也为t的n次多项式;
    \item 如果$f(t)=P(t)e^{\lambda t}$,$\lambda$为k重特征根,则特解$y_p(t)=t^k Q(t)e^{\lambda t}$,其中Q是与P次数相同的多项式,即$deg Q=degP$;
    \item 如果$f(t)=(P_1(t)\cos(\omega t)+P_2(t)\sin(\omega t))e^{\lambda t}$,$\lambda\pm i\omega$是k重特征根,则特解$y_p(t)=t^k(Q_1(t)\cos(\omega t)+Q_2(t)\sin(\omega t))e^{\lambda t}$,其中$degQ_1=degQ_2=max\{degP_1,degP_2\}$
\end{itemize}
如果确定了一组初始条件$y(t_0),y'(t_0),\cdots,y^{(n-1)}(t_0)$,则微分方程有唯
一的解,因为完全解中的特解部分是确定的,而齐次解部分的n个解对应着n个系数,只要给
定n个初始条件,就相当于给出了n个方程,可以解出所有的系数。

下面介绍一些信号与系统课程中会用到的术语。

一个用常系数线性微分方程描述的系统,就是指激励为$e(t)=f(t)$,响应为微分方程的完
全解$y(t)$的系统,齐次解$y_h(t)$相当于在激励为0时系统的响应。其中,齐次解又称为
\textbf{自由响应}(natural response),它不依赖于激励的形式,而特解又称为
\textbf{强迫响应}(forced response)。注意区分它们与\ref{sec:System}中介绍的
零状态响应、零输入响应,后者对于一般的系统都有定义,对于用微分方程描述的系统,齐
次解和特解、零输入响应和零状态响应也是不同的分类方式。

\noindent 例2.1.由微分方程及初始条件
\[y'+2y=10,y(0)=1\]
描述的系统,采用数学的解法,可以求出其特征方程$\lambda+2=0$的解$\lambda=-2$,
从而有齐次解$y_h(t)=Ce^{-2t}$,C为常数,特解$y_p(t)=5$,完全解
$y(t)=y_h(t)+y_p(t)=5+Ce^{-2t}$,带入$y(0)=1$得$5+C=1,C=-4$,因此最终的解为
\[y(t)=5-4e^{-2t}\]
采用系统响应分解,求零输入响应时,应求齐次方程满足初始条件得解,即令$y_h(0)=C=1$,
得到$C=1,y_{z.i.}(t)=e^{-2t}$;而求零状态响应时,则应该求原方程在初始条件为0向
量情况下的解,即在前面得到的完全解中令$y(0)=5+C=0$,得到$C=-5$,
$y_{z.s.}(t)=5-5e^{-2t}$,将零输入响应、零状态响应相加,又得到了原来的解
$y(t)=5-4e^{-2t}$。

可见,$y_h(t)\neq y_{z.i.}(t),y_p(t)\neq y_{z.s.}(t)$。
尽管从数学上我们先得到了齐次解和特解才进一步求出零输入响应、零状态响应,但解的这
两种划分都是有意义的,后者揭示了:响应中有一部分齐次解用来使响应满足初始条件,而
另一部分齐次解会与特解叠加得到无关初始条件、仅依赖于方程形式的解。

由于我们对输入激励建模时,常常认为激励是瞬间施加到系统上的,并且经常认为激励施加
的时间就是0时刻,我们需要区分0的左、右邻域内的系统状态,为此将激励接入之前的瞬间
系统的状态称为\textbf{$0_-$状态}或\textbf{起始状态},记为
\[y_{(k)}(0_-)=[y(0_-),y'(0-),\cdots,y^{(n-1)}(0_-)]\]
将激励接入之后的瞬间系统的状态称为\textbf{$0_+$状态}或\textbf{初始状态},记为
\[y_{(k)}(0_+)=[y(0_+),y'(0+),\cdots,y^{(n-1)}(0_+)]\]
注意此时我们是允许一些条件发生突变的。

\noindent 例2.2.RLC振荡电路中,对于电感,
$u_L=Ldi_L/dt$,施加于电感两侧的电压不会无穷大,其电流$i_L(t)$总是连续的,即
$i_L(0_-)=i_L(0_+)=i_L(0)$,而其电压则可以不连续;对于电容,$i_C=Cdu_C/dt$,
通过电容的电流不会无穷大,其电压$u_C(t)$总是连续的,即
$u_C(0_-)=u_C(0_+)=u_C(0)$,而其电流可以不连续。电感电流、电容电压不突变的这个
结果,称为换路定则。

如果激励$f(t)$不连续,并不一定意味着$y(t)$多次连续可导,但$y^{(n)}(t)$的性质突
然变差以至于$f(t)$不连续(构造这样的函数是比较困难的,函数形
式也将比较复杂),而是$y(t)$不连续,这时需要将y视为一个分布来求导。
例如$f(t)=g(t)u(t)$时,完全解中可能含有$u(t)$及其导数$\delta(t)$。

物理中的受迫振动,以及电路分析中的二阶电路,都是用二阶常系数线性微分方程描述的系
统的例子,从直观上看,如果使激励(施加于系统的力,或者电源提供的电压、电流)经过
一段时间后变为0,则响应一定也会随时间的增大而趋于0。因此,又将响应分为
\textbf{稳态响应}(steady state response)$y_{ss}(t)$\\
和\textbf{暂态响应}(transient state)$y_t(t)$,完全解中,$t\to\infty$时保留下来
的分量称为稳态响应,例如一个常数,趋于0的分量为稳态响应,例如$e^{-\lambda t}$。

由常系数线性齐次微分方程描述的系统,如果初始状态为0向量,验证其时不变性是十分简
单的。设一个系统描述为
\[a_n r^{(n)}+a_{n-1}r^{(n-1)}+\cdot +a_1 r'+a_0 r=e\]
并且对于给定的激励$e(t)$找到了完全解$r(t)$,那么\begin{align*}
    \tau_b e&=\tau_b(a_n r^{(n)}+a_{n-1}r^{(n-1)}+\cdot +a_1 r'+a_0 r)\\
    &=a_n \tau_b r^{(n)}+a_{n-1}\tau_b r^{(n-1)}+\cdot +a_1 \tau_b r'+a_0 \tau_b r\\
    &=a_n (\tau_b r)^{(n)}+a_{n-1}(\tau_b r)^{(n-1)}+\cdot +a_1 (\tau_b r)'+a_0 (\tau_b r)
\end{align*}
即激励产生延时b时,零状态响应也产生延时b。至于零输入响应,不会由于激励的改变而改
变,它不仅会破坏系统的时不变性,还会破坏系统的线性性,因为在初始状态非0时,直接
将两个完全解相加,则所得信号的初始条件变为给定初始条件的两倍,但由于零输入相应是
易于研究的,并且只要指出了零输入响应,就可以认为系统的初始状态为n维零向量来进行
研究,最后将所得的零状态响应与零输入响应叠加,因此我们约定:由常系数线性微分方程
描述的系统总是线性时不变系统,不论初始状态是否为0。从研究微分方程描述系统的线性、
时不变性的过程可以看出,提出零状态响应、零输入响应的分解是必要的,并且我们还将很
快看到这样的分解方式对于求解常系数线性微分方程的好处。

以线性时不变系统的视角来看微分方程,自然会想到求这个系统的单位脉冲响应,也就是说,
令方程右侧的激励信号为$\delta$,看所得响应$h(t)$,这样,根据上一小节中的结果,
不论方程右侧的激励$e(t)$变为何种形式,只要它与$h(t)$的卷积是有定义的,就可以直接
求出响应(或者说零状态响应)$r(t)=(h*e)(t)$,只要再叠加上零输入响应,就可以得到
任意给定初始状态下方程的解,这是一个一劳永逸的工作,在实际问题中,我们经常给出系
统的微分方程描述和初始条件,而激励则是任意的,现在就可以避开微分方程,直接计算卷
积来求得系统的响应。数学上,这个脉冲响应$h(t)$被称为\textbf{基本解}(fundamental solution),
它在偏微分方程、数学物理和线性系统理论等领域中都有广泛的应用。

那么,如何求这个单位脉冲响应呢?对于微分方程
\[a_n r^{(n)}+a_{n-1}r^{(n-1)}+\cdot +a_1 r'+a_0 r=\delta\]
要使等式右边出现$\delta$,$r(t)$中一定含有$\delta$及其导数、积分。
我们知道,$u'=\delta$,而$u(t)$的各阶积分都可以用$P(t)u(t)$来表示,其中$P(t)$
是t的多项式,因此可以假定$r(t)=f_1(t)u(t)+f_2(t)\delta(t)+f_3(t)\delta'(t)+\cdots$。
我们知道,$\delta$具有取样性质:$\delta(t)f(t)=\delta(t)f(0)$,类似地,还推导
过$\delta'$与函数的乘积:$g\delta'=g(0)\delta'-g'(0)\delta$,不难想象,$\delta$
的各阶导数与函数的乘积都是类似的(可以通过归纳法验证这一点),于是刚才的$r(t)$简
化为
\[r(t)=f(t)u(t)+a_1\delta+a_2\delta'+\cdots\]
我们知道,分布与函数乘积的求导也满足莱布尼兹法则,运用这一点,将上式带入激励取为
$\delta$的微分方程,可以通过待定系数法求出各个系数,并得到关于$f(t)$的常规的微分
方程。很明显,我们不总是需要$\delta$及其高阶导数项,为此需要研究取到$u(t)$的几阶
导数就足以完成基于待定系数法的求解,我们通过一个简单的例子来说明这个问题。

\noindent 例2.3.求微分方程
\[r''(t)+3r'(t)+2r(t)=e(t)\]
的单位脉冲响应。令$r''(t)+3r'(t)+2r(t)=\delta$,如果$r(t)$中含有$\delta'$,则
其各阶导数会因此含有$\delta$的更高阶导数,这不是我们所需要的,因此,令
$r(t)=f(t)u(t)+a\delta$,求出其一阶、二阶导数:\begin{align*}
    r'(t)&=f'(t)u(t)+f(t)\delta+a\delta'\\
    &=f'(t)u(t)+f(0)\delta+a\delta'\\
    r''(t)&=f''(t)u(t)+f'(t)\delta+f(0)\delta'+a\delta''\\
    &=f''(t)u(t)+f'(0)\delta+f(0)\delta'+a\delta''
\end{align*}
再带入原方程:\begin{align*}
    r''(t)+3r'(t)+2r(t)=&[f''(t)u(t)+f'(0)\delta+f(0)\delta'+a\delta'']\\
    &+3[f'(t)u(t)+f(0)\delta+a\delta']\\
    &+2[f(t)u(t)+a\delta]\\
    =&[f''(t)+3f'(t)+2f(t)]u(t)+[f'(0)+3f(0)+2a]\delta+[f(0)+3a]\delta'+a\delta''
\end{align*}
对比系数,得到:\begin{align*}\begin{cases}
    &f''(t)+3f'(t)+2f(t)=0\\
    &f'(0)+3f'(0)+2a=1\\
    &f(0)+3a=0\\
    &a=0
\end{cases}
\end{align*}
因此问题转化为求解微分方程
\[f''(t)+3f'(t)+2f(t)=0,f(0)=0,f'(0)=1\]
这个方程正是一开始的微分方程的齐次方程,可以立即得到它的解:
\begin{align*}
    f(t)=e^{-t}-e^{-2t}\\
    h(t)=(e^{-t}-e^{-2t})u(t)
\end{align*}
总之,只要将齐次解乘以$u(t)$,求出对应阶导数并带回原方程对比系数,即可得到单位脉
冲响应;由于单位阶跃响应形如$h(t)=y_h(t)u(t)$,满足$h(t)=0,t<0$,微分方程描述的
系统还具有因果性。

我们还可以对以上结果做一些推广:求解微分方程
\[\sum_{n=0}^{N}a_n r^{(n)}(t)=\sum_{m=0}^{M}b_m \delta^{(m)}\]
当$r(t)$的最高次导的阶数不小于$\delta$的最高阶导阶数,即$N\geq M$时,解为
$y_h(t)u(t)$;当$r(t)$的最高次导的阶数大于$\delta$的最高阶导阶数,即$N<M$时,
解为\[y_h(t)u(t)+\sum_{k=0}^{M-N}c_k \delta^{(k)}\]
换言之,需要利用方程左侧的最高阶导,让方程右侧所需的$\delta$的最高阶导数项出现。
这个结果可以求出以下形式的系统的脉冲响应:
\[\sum_{n=0}^{N}a_n r^{(n)}(t)=\sum_{m=0}^{M}b_m e^{(m)}(t)\]
这同样时微分方程描述的系统,只不过从数学上我们没有必要考虑这种形式的方程,只要把
方程右侧看作一个新的非齐次项,它与其他微分方程没有区别,但当系统用这种方式描述时,
求解其脉冲响应就显得十分重要了。当然,正因为这种方程本质上还是一个非齐次的常系数
线性微分方程,我们仍然可以用基本解的方法来求得脉冲响应,只需要在求出基本解后,将
基本解与方程右侧做卷积,不过实际上记忆$\delta$的各阶导数的卷积特性并不是很有必要
,通过待定系数法求解往往更快。


\end{document}